\include{header}

\title[Tutorium 5]{GBI Tutorium Nr. $2^5$}
\subtitle{Tutorium 5}
\date{21. November 2012}

% Bibliography



\begin{document}

	%title page
	\begin{frame}
		\titlepage
	\end{frame}

	%table of contents
	\begin{frame}{Outline/Gliederung}
		\tableofcontents
	\end{frame}
	
	
	\section{\"Ubungsblatt 4}
	\begin{frame} {Übungsblatt 4}
		
	\end{frame}	
		
	
	
	
	\section{Wiederholung} 
	\begin{frame} {Wiederholung - Quiz}
		\begin{itemize}
			\item $X=X$ ist eine Schleifeninvariante! 
			\only<2-> {\color{darkgreen}$\surd$}\\
			\color{black}
			\item $(L_1 \cdot L_2)^* = L_1^* \cdot L_2^*$
			\only<3-> {\color{red}$X$}\\
			\color{black}
			\item $f(x) = x^3-x^2$ ist rechtstotal für $x, f(x) \in \mathbb{R}$.
			\only<4-> {\color{darkgreen}$\surd$}\\
			\color{black}
			\item Eine bijektive Relation ist eine Funktion.
			\only<5-> {\color{red}$X$}\\
			\color{black}
			\item Wenn $f: A \rightarrow B$ injektiv $\Rightarrow f^{-1}$ ist surjektiv.
			\only<6-> {\color{red}$X$}\\
			\color{black}
			\item $A\Rightarrow B \Leftrightarrow \neg B \Rightarrow \neg A$
			\only<7-> {\color{darkgreen}$\surd$}\\
			\color{black}
			\item $(\forall x \exists y \; | \; f(x,y)) \Leftrightarrow (\exists y \forall x \; | \; f(x,y)))$
			\only<8-> {\color{red}$X$}\\
		\end{itemize}
	\end{frame}
	
	
	
	\begin{frame} {Wiederholung - Aufgaben}
		\begin{block}{Relationen}
			Es sei A die Menge aller Kinobesucher in einer Vorstellung 
			und B die Menge aller Sitzplätze. 
			Die Abbildung f ordnet den Kinobesuchern die Sitzplätze zu:\\
			\hspace{50pt}	$ f : A \rightarrow B $
			\begin{itemize}
				\pause
				\item Was wünschen sich die Kinobesucher: 
				Eine injektive, surjektive oder bijektive Abbildung 
				auf die Sitzplätze? Was wünscht sich der Kinobesitzer?
				
				\pause
				\item Erklären Sie jeweils, was es im Kino bedeutet, wenn $f$ linkstotal, linkseindeutig, rechtstotal, rechtseindeutig ist.
				
				\pause
				\item In dieser Teilaufgabe nehmen wir an, 6 Kinobesucher 
				besuchten ein Kino mit 8 Plätzen. 
				Zeichnen Sie eine injektive Abbildung $f$. 
				Wie viele injektive Abbildungen gibt es?
			\end{itemize}
		\end{block}
	\end{frame}
	
	
	
	\begin{frame} {Wiederholung}
		\begin{block}{Schleifeninvarianz}
			Gegeben sei folgender Algorithmus:\\
			\begin{algorithmic}
				\State $x \gets a;$
				\State $y \gets b;$
				\State $p \gets 0;$
				\While{$x > 0$}
					\State $p \gets p + y$
					\State $x \gets x - 1$
				\EndWhile
			\end{algorithmic}
			
			\begin{itemize}
				\pause
				\item Was macht dieser Algorithmus?
				
				\pause
				\item Stellen Sie eine Schleifeninvariante über alle 
				Variablen auf
				
				\pause
				\item Beweisen Sie Ihre Schleifeninvariante
			\end{itemize}						
		\end{block}
	\end{frame}
	
	
	
	\section{Relationen 2}
	\begin{frame}{Relationen 2}
		\begin{itemize}
			\item Wie war eine Relation Definiert?
			
			\pause
			\item 
		\end{itemize}
	\end{frame}
	
	
	\subsection{Reflexivität}
	\begin{frame}{Reflexivität}
		\begin{block}{Definition}
			$\forall x \in M \;|\; (x,x) \in R$\\
			$\Rightarrow R \subseteq M \times M$\\
			\pause
			was sagt uns diese Definition?
		\end{block}
	\end{frame}
	
	
	\subsection{Transitivität}
	\begin{frame}{Transitivität}
		\begin{block}{Definition}
			$\forall x,y,z \;|\; ((x,y) \in R \land (y,z) \in R) \Rightarrow (x,z) \in R$\\
		\end{block}
	\end{frame}
	
	
	
	\section{Fragen}
	\begin{frame} {Fragen}
		\begin{itemize}
			\item Fragen zum Stoff?
			\item Fragen zum n\"achsten \"Ubungsblatt?
			\item Generelle Fragen?
			\item Feedback?
		\end{itemize}
	\end{frame}

		
	\begin{frame} {EOF}
		\begin{center}
			\includegraphics[scale=0.4]{graphics/eof4.png}\\
			\tiny $source: http://imgs.xkcd.com/comics/michael_phelps.png$
		\end{center}
	\end{frame}

\end{document}
