\include{header}

\title[Tutorium 5]{GBI Tutorium Nr. $2^5$}
\subtitle{Tutorium 5}
\date{21. November 2012}

% Bibliography



\begin{document}

	%title page
	\begin{frame}
		\titlepage
	\end{frame}

	%table of contents
	\begin{frame}{Outline/Gliederung}
		\tableofcontents
	\end{frame}
	
	
	\section{\"Ubungsblatt 4}
	\begin{frame} {Übungsblatt 4}
		
	\end{frame}	
		
	
	
	
	\section{Wiederholung} 
	\begin{frame} {Wiederholung - Quiz}
		\begin{itemize}
			\item $X=X$ ist eine Schleifeninvariante! 
			\only<2-> {\color{darkgreen}$\surd$}\\
			\color{black}
			
			\item $A \Rightarrow B \Leftrightarrow \neg A \lor B$
			\only<2-> {\color{darkgreen}$\surd$}\\
			\color{black}
		\end{itemize}
	\end{frame}
	
	
	
	\begin{frame} {Wiederholung - Aufgaben}
		\begin{block}{Relationen}
			Es sei A die Menge aller Kinobesucher in einer Vorstellung 
			und B die Menge aller Sitzplätze. 
			Die Abbildung f ordnet den Kinobesuchern die Sitzplätze zu:\\
			\hspace{50pt}	$ f : A \rightarrow B $
			\begin{itemize}
				\pause
				\item Was wünschen sich die Kinobesucher: 
				Eine injektive, surjektive oder bijektive Abbildung 
				auf die Sitzplätze? Was wünscht sich der Kinobesitzer?
				
				\pause
				\item Erklären Sie jeweils, was es im Kino bedeutet, wenn $f$ linkstotal, linkseindeutig, rechtstotal, rechtseindeutig ist.
				
				\pause
				\item In dieser Teilaufgabe nehmen wir an, 6 Kinobesucher 
				besuchten ein Kino mit 8 Plätzen. 
				Zeichnen Sie eine injektive Abbildung $f$. 
				Wie viele injektive Abbildungen gibt es?
			\end{itemize}
		\end{block}
	\end{frame}
	
	
	
	\begin{frame} {Wiederholung}
		\begin{block}{Schleifeninvarianz}
			Gegeben sei folgender Algorithmus:\\
			\begin{algorithmic}
				\State $x \gets a;$
				\State $y \gets b;$
				\State $p \gets 0;$
				\While{$x > 0$}
					\State $p \gets p + y$
					\State $x \gets x - 1$
				\EndWhile
			\end{algorithmic}
			
			\begin{itemize}
				\pause
				\item Was macht dieser Algorithmus?
				
				\pause
				\item Stellen Sie eine Schleifeninvariante über alle 
				Variablen auf
				
				\pause
				\item Beweisen Sie Ihre Schleifeninvariante
			\end{itemize}						
		\end{block}
	\end{frame}
	
	
	
	\section{Relationen 2}
	\begin{frame}{Relationen 2}
		\begin{itemize}
			\item Wie war eine Relation Definiert?
			
			\pause
			\item Was bedeutet $xRy$?
			
			\pause
			\item Wie lassen sich Relationen darstellen?
			
			\pause
			\item Welche Besonderheiten haben Relationen?
			
			\pause
			\item gibt es weitere Besonderheiten?
		\end{itemize}
	\end{frame}
	
	
	\subsection{Reflexivität}
	\begin{frame}{Reflexivität}
		\begin{block}{Definition}
			$\forall x \in M \;|\; (x,x) \in R$\\
			$\Rightarrow R \subseteq M \times M$\\
			\pause
			\vspace{10pt}
			Was sagt uns diese Definition?
		\end{block}
		
		\begin{exampleblock}{Beispiel}
			\begin{itemize}
				\item Die $\leq$ Relation ist reflexiv\\
					\visible<3->{\hspace{30pt} $\hookrightarrow$ Warum?}
					
				\pause
				\pause
				\item Ist die Relation: $R = \{(x,y)\in M \times M \;|\; y = x^2\} $ reflexiv?
			\end{itemize}
		\end{exampleblock}
	\end{frame}
	
	
	\subsection{Transitivität}
	\begin{frame}{Transitivität}
		\begin{block}{Definition}
			$\forall x,y,z \;|\; ((x,y) \in R \land (y,z) \in R) \Rightarrow (x,z) \in R$\\
		\end{block}
		
		\begin{exampleblock}{Beispiel}
			\begin{itemize}
				\item Die $\leq$ Relation ist auch transitiv.
				
				\pause
				\item 
			\end{itemize}
		\end{exampleblock}
	\end{frame}
	
	
	\subsection{Symmetrie}
	\begin{frame}{Symmetrie}
		\begin{block}{Definition}
			$\forall x, y \in M \; | \; (x,y) \in R \rightarrow (y,x) \in R$\\
			$\Rightarrow R \subseteq M \times M$\\
			\pause
			\vspace{10pt}
			Wie sieht eine solche Relation grafisch aus?
		\end{block}
		
		\begin{exampleblock}{abc}
			\begin{itemize}
				\item Die $=$ Relation ist symetrisch\\
				
				\pause
				\item Ist die Relation: 
					$
					R = f(x) =
					\begin{cases}
					x + 1 & \text{falls x gerade}\\
					x - 1 & \text{falls x ungerade}
					\end{cases}
					$
					\\ mit $x\in \mathbb{N}_0$ und $f(x) \in \mathbb{N}_0$ symmetrisch?
			\end{itemize}
		\end{exampleblock}
	\end{frame}
	
	
	\begin{frame}{Aufgaben}
		abc
	\end{frame}
	
	
	\section{Kontextfreie Grammatiken}
	\begin{frame}{Definition}
		\begin{block}{Definition}
        	Die Menge $G = G \left( N, T, S, P \right)$ nennen wir 
        	\textbf{Kontextfreie Grammatik}.
    	\end{block}
    	\pause
        \begin{block}{Was ist was?}
            \begin{itemize}
                \item \visible<3->{$N$: \textbf{Nichtterminalsymbol}}
                \item \visible<4->{$T$: \textbf{Terminalsymbol}}
                \item \visible<5->{$S$: \textbf{Startsymbol}}
                \item \visible<6->{$P$: \textbf{Produktionsmenge}}
            \end{itemize}
    	\end{block}
	\end{frame}
	
	\begin{frame}{Sinn?}
		Für was brauchen wir kontextfreie Grammatiken?		
		\begin{itemize}
			\item
		\end{itemize}
	\end{frame}
	
	
	\begin{frame}{Ableitung}
    	\begin{block}{Definition}
        	Als Ableitung wird in der theoretischen Informatik der Vorgang 
        	bezeichnet, ein Wort nach den Regeln einer formalen 
        	Grammatik zu erzeugen.
   		\end{block}
   		
    	Wir schreiben: $w \Rightarrow^i v$, wenn von der Ableitung 
    	von $v$ aus $w$ $i$ Ableitungsschritte liegen ($i \in \mathbb{N}$).
    
    	\pause
   		\visible<2->{
    		\begin{alertblock}{Vorsicht}
        		\begin{align*}
            		\Rightarrow \neq \rightarrow
        		\end{align*}
        		
        		\begin{itemize}
            		\item $\Rightarrow$ ist die Relation der Ableitung
            	\item $\rightarrow$ ist die Relation der Produktion ($\in P$)
        		\end{itemize}
    		\end{alertblock}
    	}
	\end{frame}

	\begin{frame}{Ableitung}
    	\begin{block}{Frage}
        	Was stimmt? Es ist $w_1, w_2 \in N \cup P$.
        	
        	\begin{itemize}
            	\item $w_1 \rightarrow w_2$, daraus folgt $w_1 \Rightarrow w_2$
            	\item $w_1 \Rightarrow w_2$, daraus folgt $w_1 \rightarrow w_2$
        	\end{itemize}
    	\end{block}
	\end{frame}

	\begin{frame}{Sprache der kontextfreien Grammatik}
    	\begin{block}{Definition}
        	Sei $G$ eine kontextfreie Grammatik. 
        	Dann bezeichnen wir die Sprache $L = L\left(G\right)$ mit 
        
        	\begin{align*}
            	L = \left\{ w\in T^* \big| S \Rightarrow^* w\right\}
        	\end{align*}
    	\end{block}
    
    	\pause
    	\visible<2->{
    		\begin{block}{Was ist $\Rightarrow^*$?}
        		Mit $\Rightarrow^*$ ist die \emph{reflexiv-transitive Hülle} 
        		der Ableitungsrelation gemeint.
    		\end{block}
    	}
	\end{frame}	
	
	
	
	\section{Aufgaben}
	\begin{frame}{Aufgaben}
		
	\end{frame}
	
	
	
	\section{Fragen}
	\begin{frame} {Fragen}
		\begin{itemize}
			\item Fragen zum Stoff?
			\item Fragen zum n\"achsten \"Ubungsblatt?
			\item Generelle Fragen?
			\item Feedback?
		\end{itemize}
	\end{frame}

		
	\begin{frame} {EOF}
		\begin{center}
			\includegraphics[scale=0.5]{graphics/eof5.png}\\
			\tiny $source: http://imgs.xkcd.com/comics/home\_organization.png$
		\end{center}
	\end{frame}

\end{document}
