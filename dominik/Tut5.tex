\include{header}

\title[Tutorium 5]{GBI Tutorium Nr. $2^5$}
\subtitle{Tutorium 5}
\date{21. November 2012}

% Bibliography



\begin{document}

	%title page
	\begin{frame}
		\titlepage
	\end{frame}

	%table of contents
	\begin{frame}{Outline/Gliederung}
		\tableofcontents
	\end{frame}
	
	
	\section{\"Ubungsblatt 4}
	\begin{frame} {Übungsblatt 4}
		
	\end{frame}	
		
	
	
	
	\section{Wiederholung} 
	\begin{frame} {Wiederholung - Quiz}
		\begin{itemize}
			\item $A^*$ ist eine formale Sprache! 
			\only<2-> {\color{darkgreen}$\surd$}\\
			\color{black}
			\item $(L_1 \cdot L_2)^* = L_1^* \cdot L_2^*$
			\only<3-> {\color{red}$X$}\\
			\color{black}
			\item $f(x) = x^3-x^2$ ist rechtstotal für $x, f(x) \in \mathbb{R}$.
			\only<4-> {\color{darkgreen}$\surd$}\\
			\color{black}
			\item Eine bijektive Relation ist eine Funktion.
			\only<5-> {\color{darkgreen}$\surd$}\\
			\color{black}
			\item Wenn $f: A \rightarrow B$ injektiv $\Rightarrow f^{-1}$ ist surjektiv.
			\only<6-> {\color{red}$X$}\\
			\color{black}
			\item $A\Rightarrow B \Leftrightarrow \neg B \Rightarrow \neg A$
			\only<7-> {\color{darkgreen}$\surd$}\\
			\color{black}
			\item $(\forall x \exists y \; | \; f(x,y)) \Leftrightarrow (\exists y \forall x \; | \; f(x,y)))$
			\only<8-> {\color{red}$X$}\\
		\end{itemize}
	\end{frame}
	
	
	
	\begin{frame} {Wiederholung - Aufgaben}
		\begin{itemize}
			\item Schreiben sie die Injektivität als prädikatenlogische Formel.
			\pause
			\item Es sei $L \subseteq A^*$ eine formale Sprache. Beweisen oder widerlegen Sie:\\
			$L^+ \cdot L^+ \subseteq L^+$
			\pause
			\item Geben Sie eine formale Sprache $L$ über dem Alphabet $A=\{0,1\}$ an, für welche gilt:\\
				$L$ enthält nur Wörter gerader Parität. 
				Die Parität eines Wortes ist gerade, wenn eine gerade Anzahlen von 1en enthalten sind.
		\end{itemize}
	\end{frame}
	
	
	
	\begin{frame} {Wiederholung}
		\begin{block}{Induktion}
			Alice und Bob feiern ihren Hochzeitstag. Auf ihrer Party befinden sich $n \in \mathbb{N}_+$ Paare.
			Dabei begrüßen sich alle Paare mit Ausnahme des eigenen Partners.\\
			\vspace{10pt}
			a) Geben Sie die Anzahl der Begrüßungen $x_i$ für $i \in \{1,2,3,4,5\}$ Paare an.\\
			\vspace{10pt}
			b) Stellen Sie für $x_n$ eine geschlossene Formel (d.h. einen arithmetischen Ausdruck, in dem nur Zahlen, n und die Grundrechenarten vorkommen) auf.\\
			\vspace{10pt}
			c) Beweisen Sie Ihre Aussage aus Teilaufgabe b) durch vollständige Induktion
		\end{block}
	\end{frame}
	
	
	
	\section{Fragen}
	\begin{frame} {Fragen}
		\begin{itemize}
			\item Fragen zum Stoff?
			\item Fragen zum n\"achsten \"Ubungsblatt?
			\item Generelle Fragen?
			\item Feedback?
		\end{itemize}
	\end{frame}

		
	\begin{frame} {EOF}
		\begin{center}
			\includegraphics[scale=0.4]{graphics/eof4.png}\\
			\tiny $source: http://imgs.xkcd.com/comics/michael_phelps.png$
		\end{center}
	\end{frame}

\end{document}
