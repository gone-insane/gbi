\include{header}

\title[Tutorium 5]{GBI Tutorium Nr. $2^5$}
\subtitle{Tutorium 5}
\date{21. November 2012}

% Bibliography



\begin{document}

	%title page
	\begin{frame}
		\titlepage
	\end{frame}

	%table of contents
	\begin{frame}{Outline/Gliederung}
		\tableofcontents
	\end{frame}
	
	
	\section{\"Ubungsblatt 4}
	\begin{frame} {Übungsblatt 4}
		\begin{center}
			\Huge ?
		\end{center}
	\end{frame}	
		
	
	
	
	\section{Wiederholung} 
	\begin{frame} {Wiederholung - Quiz}
		\begin{itemize}
			\item $X=X$ ist eine Schleifeninvariante! 
			\only<2-> {\color{darkgreen}$\surd$}\\
			\color{black}
			
			\item $A \Rightarrow B \Leftrightarrow \neg A \lor B$
			\only<2-> {\color{darkgreen}$\surd$}\\
			\color{black}
		\end{itemize}
	\end{frame}
	
	
	
	\begin{frame} {Wiederholung - Aufgaben}
		\begin{block}{Relationen}
			Es sei A die Menge aller Kinobesucher in einer Vorstellung 
			und B die Menge aller Sitzplätze. 
			Die Abbildung f ordnet den Kinobesuchern die Sitzplätze zu:\\
			\hspace{50pt}	$ f : A \rightarrow B $
			\begin{itemize}
				\pause
				\item Was wünschen sich die Kinobesucher: 
				Eine injektive, surjektive oder bijektive Abbildung 
				auf die Sitzplätze? Was wünscht sich der Kinobesitzer?
				
				\pause
				\item Erklären Sie jeweils, was es im Kino bedeutet, wenn $f$ linkstotal, linkseindeutig, rechtstotal, rechtseindeutig ist.
				
				\pause
				\item In dieser Teilaufgabe nehmen wir an, 6 Kinobesucher 
				besuchten ein Kino mit 8 Plätzen. 
				Zeichnen Sie eine injektive Abbildung $f$. 
				Wie viele injektive Abbildungen gibt es?
			\end{itemize}
		\end{block}
	\end{frame}
	
	
	
	\begin{frame} {Wiederholung}
		\begin{block}{Schleifeninvarianz}
			Gegeben sei folgender Algorithmus:\\
			\begin{algorithmic}
				\State $x \gets a;$
				\State $y \gets b;$
				\State $p \gets 0;$
				\While{$x > 0$}
					\State $p \gets p + y$
					\State $x \gets x - 1$
				\EndWhile
			\end{algorithmic}
			
			\begin{itemize}
				\pause
				\item Was macht dieser Algorithmus?
				
				\pause
				\item Stellen Sie eine Schleifeninvariante über alle 
				Variablen auf
				
				\pause
				\item Beweisen Sie Ihre Schleifeninvariante
			\end{itemize}						
		\end{block}
	\end{frame}
	
	
	
	\section{Relationen 2}
	\begin{frame}{Relationen 2}
		\begin{itemize}
			\item Wie war eine Relation Definiert?
			
			\pause
			\item Was bedeutet $xRy$?
			
			\pause
			\item Wie lassen sich Relationen darstellen?
			
			\pause
			\item Welche Besonderheiten haben Relationen?
			
			\pause
			\item gibt es weitere Besonderheiten?
		\end{itemize}
	\end{frame}
	
	
	\subsection{Reflexivität}
	\begin{frame}{Reflexivität}
		\begin{block}{Definition}
			$\forall x \in M \;|\; (x,x) \in R$\\
			$\Rightarrow R \subseteq M \times M$\\
			\pause
			\vspace{10pt}
			Was sagt uns diese Definition?
		\end{block}
		
		\begin{exampleblock}{Beispiel}
			\begin{itemize}
				\item Die $\leq$ Relation ist reflexiv\\
					\visible<3->{\hspace{30pt} $\hookrightarrow$ Warum?}
					
				\pause
				\pause
				\item Ist die Relation: $R = \{(x,y)\in M \times M \;|\; y = x^2\} $ reflexiv?
			\end{itemize}
		\end{exampleblock}
	\end{frame}
	
	
	\subsection{Transitivität}
	\begin{frame}{Transitivität}
		\begin{block}{Definition}
			$\forall x,y,z \;|\; ((x,y) \in R \land (y,z) \in R) \Rightarrow (x,z) \in R$\\
		\end{block}
		
		\begin{exampleblock}{Beispiel}
			\begin{itemize}
				\item Die $\leq$ Relation ist auch transitiv.\\
				siehe Axiom 12 der reellen Zahlen.
				
				\pause
				\item Ist $M \times M$ transitiv?
			\end{itemize}
		\end{exampleblock}
	\end{frame}
	
	
	\subsection{Symmetrie}
	\begin{frame}{Symmetrie}
		\begin{block}{Definition}
			$\forall x, y \in M \; | \; (x,y) \in R \rightarrow (y,x) \in R$\\
			$\Rightarrow R \subseteq M \times M$\\
			\pause
			\vspace{10pt}
			Wie sieht eine solche Relation grafisch aus?
		\end{block}
		
		\begin{exampleblock}{Beispiel}
			\begin{itemize}
				\item Die $=$ Relation ist symetrisch\\
				
				\pause
				\item Ist die Relation: 
					$
					R = f(x) =
					\begin{cases}
					x + 1 & \text{falls x gerade}\\
					x - 1 & \text{falls x ungerade}
					\end{cases}
					$
					\\ mit $x\in \mathbb{N}_0$ und $f(x) \in \mathbb{N}_0$ symmetrisch?
			\end{itemize}
		\end{exampleblock}
	\end{frame}
	
	\begin{frame}{Äquivalenzrelationen}
		\begin{definition}
    	Sei R eine homogene Relation über M. $x, y, z \in M$.\\
    	Hat R folgende Eigenschaften:
			\begin{description}
				\item[reflexiv] $x R x$
				\item[transitiv] $x R y \wedge y R z \Rightarrow x R z$
				\item[symmetrisch] $x R y \Rightarrow y R x$
			\end{description}
			So heißt R eine \emph{Äquivalenzrelation}.
		\end{definition}
	\end{frame}
	
	
	\subsection{Produkt}
	\begin{frame}{Produkt}
		\begin{block}{Definition}
			Auch Relationen kann man Verketten, so gilt:\\
			Sei $f = A \rightarrow B$ und $ g = B$\\
			$\rightarrow C$ $\Rightarrow g \circ f = g(f(x)) = A \rightarrow C$
		\end{block}
	\end{frame}
	
	\subsection{Potenzen}
	\begin{frame}{Potenzen}
		\begin{block}{Definition}
			gilt: $R \subseteq M \times M$\\
			Dann kann man $R$ auch Potenzieren:\\
			\begin{itemize}
			\pause
			\item $R \circ R = R^2$\\
			
			\pause
			\item $R^0 = I_M = \{(x,x) \; | \; x \in M \}$\\
			
			\pause
			\item $R^{i+1} = R^i \circ R$
			
			\end{itemize}
		\end{block}
		
		\pause
		\begin{block}{Reflexiv-Transitive-Hülle}
			$R^* = \bigcup_\mathrm{i = 0}^\infty R^i$\\
			\vspace{10pt}
			\visible<6>{$R^*= R^0 \cup R^1 \cup R^2 \cup \dots \cup R^\infty$}
		\end{block}
	\end{frame}
	
	
	\begin{frame}{Beispiel (by Patrick Niklaus)}
		\begin{exampleblock}{Freunde im Netzwerk}
    		\small
			Sei $M = \{ Gertrud, Holger, Lars, Katja, Martin, Nina \}$ eine 
			Menge von Nutzern.\\
			$R \subseteq M \times M $ sei die 
			"`ist-befreundet-mit"'-Relation.
				
			\begin{itemize}
				\item $R = \{(Martin,Holger),(Lars,Katja),(Nina,Holger),$ \\
				$(Gertrud,Holger), (Katja, Nina) \} \bigcup \{$
				{dazu sym. Tupel}$\}$ \pause
					
				\item $R^0=\{ (Martin,Martin), ..., (Holger,Holger) \}$ \pause
				
				\item $R^1=R$ "'Freundschaft 1. Grades."' \pause
				
				\item $R^2=\{(Martin,Nina),(Martin,Gertrud),(Martin,Martin),$ \\
				
				$(Lars,Nina), (Lars,Lars), (Nina,Gertrud),(Nina,Martin),$ \\		
				$(Nina,Nina), (Nina,Lars), (Katja,Katja), (Katja,Holger), $ \\
				$(Gertrud,Gertrud), (Gertrud,Martin), (Gertrud,Nina), $ \\
				$(Holger,Holger), (Holger,Katja)\}$ "'Freundschaft 2. Grades"' 		
				\pause
				
				\item $R^*=?$ "'Gibt es eine Verbindung durch 
				Freunde beliebigen Grades?"'
	%\item wegen Symmetrie und Unsinnigkeit der Reflexivität, entfällt Einiges 
	% Funktionen sind kommutativ, Relationen symmetrisch
			\end{itemize}
    	
		\end{exampleblock}
	\end{frame}
	
	
	
	\begin{frame}{Aufgaben}
  		\begin{exampleblock}{Reflexiv-transitive Hülle}
    		Bestimmt die reflexiv-transitive Hülle der Relationen.\\
    		$M := \{1, 2, 3, 4\}, R_i \subset M \times M$
    		\begin{enumerate}
      			\item $R_1 := \{(1, 2), (1, 4), (2, 3)\}$
      			\item $R_2 := \{(1, 1), (2, 2)\}$
      			\item $R_3 := \{(1, 2), (3, 4), (4, 2)\}$
    		\end{enumerate}
  		\end{exampleblock}
  		\pause
  		
  		\invisible<1>{\begin{exampleblock}{Lösungen}
    	\begin{enumerate}
      	\item $R_1^* := \{(1, 1), (1, 2), (1, 3), (1, 4), (2, 2), (2, 3), (3, 3), (4, 4)\}$
      	\item $R_2^* := \{(1, 1), (2, 2), (3, 3), (4, 4)\}$
      	\item $R_3^* := \{(1, 1), (1, 2), (2, 2), (3, 2), (3, 3), (3, 4), (4, 2), (4, 4)\}$
    	\end{enumerate}
    \end{exampleblock}}
\end{frame}
	
	
	\section{Kontextfreie Grammatiken}
	\begin{frame}{Definition}
		\begin{block}{Definition}
        	Die Menge $G = G \left( N, T, S, P \right)$ nennen wir 
        	\textbf{Kontextfreie Grammatik}.
    	\end{block}
    	\pause
        \begin{block}{Was ist was?}
            \begin{itemize}
                \item \visible<3->{$N$: Nichtterminalsymbol}
                \item \visible<4->{$T$: Terminalsymbol $(T \cap N = \emptyset)$}
                \item \visible<5->{$S$: Startsymbol $(S \subseteq N)$}
                \item \visible<6->{$P$: Produktionsmenge}
            \end{itemize}
    	\end{block}
	\end{frame}
	
	\begin{frame}{Sinn?}
		Für was brauchen wir kontextfreie Grammatiken?		
		\begin{itemize}
			\pause
			\item Dienen dazu eine Grammatik zu beschreiben
			
			\pause
			\item Mit ihnen kann man Wörter einer Grammatik ableiten
		\end{itemize}
	\end{frame}
	
	
	\begin{frame}{Ableitung}
    	\begin{block}{Definition}
        	Als Ableitung wird in der theoretischen Informatik der Vorgang 
        	bezeichnet, ein Wort nach den Regeln einer formalen 
        	Grammatik zu erzeugen.
   		\end{block}
   		
    	
    	Wir schreiben: \pause
    	\begin{itemize}
        	\item $w \Rightarrow v$, wenn von der Ableitung von $v$, aus $w$, 
        	genau $1$ Ableitungsschritt liegt.
        	
        	\pause
        	\item $w \Rightarrow^i v$, wenn von der Ableitung von $v$, aus $w$, 
        	$i$ Ableitungsschritte liegen ($i \in \mathbb{N}$).
        	
        	\pause
        	\item $w \Rightarrow^* v$, wenn von der Ableitung von $v$, aus $w$, 
        	beliebig viele Ableitungsschritte liegen.
    	\end{itemize}
	\end{frame}
	
	
	\begin{frame}{Ableitung}
    	\begin{alertblock}{Vorsicht}
        	\begin{align*}
            \Rightarrow \neq \rightarrow
        	\end{align*}
        	
        	\begin{itemize}
        		\item $\Rightarrow$ ist die Relation der Ableitung
       	    	\item $\rightarrow$ ist die Relation der Produktion ($\in P$)
			\end{itemize}
		\end{alertblock}
	\end{frame}

	\begin{frame}{Ableitung}
    	\begin{block}{Frage}
        	Was stimmt? Es ist $w_1, w_2 \in N \cup P$.
        	
        	\begin{itemize}
            	\item $w_1 \rightarrow w_2$, daraus folgt $w_1 \Rightarrow w_2$
            	\item $w_1 \Rightarrow w_2$, daraus folgt $w_1 \rightarrow w_2$
        	\end{itemize}
    	\end{block}
	\end{frame}

	\begin{frame}{Sprache der kontextfreien Grammatik}
    	\begin{block}{Definition}
        	Sei $G$ eine kontextfreie Grammatik. 
        	Dann bezeichnen wir die Sprache $L = L\left(G\right)$ mit 
        
        	\begin{align*}
            	L = \left\{ w\in T^* \big| S \Rightarrow^* w\right\}
        	\end{align*}
    	\end{block}
    
    	\pause
    	\visible<2->{
    		\begin{block}{Was ist $\Rightarrow^*$?}
        		Mit $\Rightarrow^*$ ist die \emph{reflexiv-transitive Hülle} 
        		der Ableitungsrelation gemeint.
    		\end{block}
    	}
	\end{frame}	
	
	
	
	\section{Aufgaben}
	\begin{frame}{Aufgaben}
    	\begin{exampleblock}{Fragen}
        	\begin{enumerate}
          		\item Gibt es Grammatiken für die gilt: $L(G) = \{\}$?
          		
          		\visible<2->{
          		\item Welche Sprache erzeugt: 
          		$G_1 := (\{X\}, \{0\}, X, \{X \rightarrow X\})$}
          		
          		\visible<3->{
          		\item Ist 
          		$G_2 := (\{X\}, \{a, b\}, X, \{X \rightarrow \varepsilon\})$ 
          		eine gültige Grammatik?}
          		
        	\end{enumerate}
    	\end{exampleblock}
    	
    	\begin{exampleblock}{Lösung}
        	\begin{enumerate}
            	\visible<2->{\item Ja z.B. $G = (\{X\}, \{0\}, X, \{\})$}
            	\visible<3->{\item $L(G_1) = \{\}$}
            	\visible<4->{\item Ja und $L(G_2) = \{\varepsilon\}$}
        	\end{enumerate}
    	\end{exampleblock}
	\end{frame}

	
	\begin{frame}{Aufgaben}
  		\begin{exampleblock}{Welche Sprachen erzeugen folgende Grammatiken.}
    		\begin{enumerate}
      			\item $G_1 := (\{X, Y\}, \{a, b\}, X, \{X \rightarrow aY 
      			| \varepsilon, Y \rightarrow bX\})$
      			
      			\item $G_2 := (\{X, Y, Z\}, \{a, b, c\}, X,$\\
             	$\{X \rightarrow Ya | Yb | Yc, Y \rightarrow ZZY 
             	| \varepsilon, Z \rightarrow a | b | c\})$
    		\end{enumerate}
  		\end{exampleblock}
  		
  		\begin{exampleblock}
  			{Gebt eine jeweils Grammatik an für die gilt $L(G) = L_i$:}
        	\begin{enumerate}
      			\item $L_1 := \{ab, cd\}^+ \cdot \{a, c\}^2$
      			\item $A := \{0, 1\}$, $L_2 := \{w  \in A^*
      			| Num_0(w) = Num_1(w)\}$
    		\end{enumerate}
  		\end{exampleblock}
	\end{frame}


	\begin{frame}{Bonus-Aufgabe (by Patrick Niklaus)}
  		\begin{exampleblock}{Aufgabe}
    		Konstruiert eine Grammatik die alle E-Mail-Adresse 
    		aus den Buchstaben {a, b, c} erzeugt.\\
    		
    		\emph{Hinweis}: $T := \{a, b, c, @, ., \_\}$
    		
  		\end{exampleblock}\pause
  			
  		\invisible<1>{
  		\begin{exampleblock}{Lösung}
    		$G = (N, T, S, P)$
    		\begin{itemize}
      			\item $N = \{S, A, B, C\}$
      			\item $T = \{a, b, c, ., \_, @\}$
      			\item $S = S$
      			\item $P = \{S \rightarrow AB@C.C,$\\ 
      			\hspace{28pt}$A \rightarrow a|b|c,$\\ 
      			\hspace{28pt}$B \rightarrow AB|\_B|.AB|\varepsilon ,$\\ 
      			\hspace{28pt}$C \rightarrow AC | \varepsilon\}$
    		\end{itemize}
    	\end{exampleblock}}
	\end{frame}



	\begin{frame}{Aufgabe: Winter 2008/2009}
    	\begin{exampleblock}{Aufgabe}
        	\begin{itemize}
            	\item Geben Sie eine kontextfreie Grammatik 
                	\begin{align*}
                    	G = \left( N,\left\{ a,b\right\} ,S,P\right) 
                	\end{align*}
                	
                	an, für die $L\left( G\right) $ 
                	die Menge aller Palindrome über dem Alphabet 
                	$\left\{ a,b\right\} $ ist. 
                	
            	\item Geben Sie eine Ableitung der Wörter 
            	$baaab$ und $abaaaba$ aus dem Startsymbol Ihrer Grammatik an. 
            
            	\item Beweisen Sie, dass Ihre Grammatik jedes Palindrom 
            	über dem Alphabet $\left\{ a, b\right\} $ erzeugt.
        	\end{itemize}
    	\end{exampleblock}
	\end{frame}
	
	\section{Fragen}
	\begin{frame} {Fragen}
		\begin{itemize}
			\item Fragen zum Stoff?
			\item Fragen zum n\"achsten \"Ubungsblatt?
			\item Generelle Fragen?
			\item Feedback?
		\end{itemize}
	\end{frame}

		
	\begin{frame} {EOF}
		\begin{center}
			\includegraphics[scale=0.5]{graphics/eof5.png}\\
			\tiny $source: http://imgs.xkcd.com/comics/home\_organization.png$
		\end{center}
	\end{frame}

\end{document}
