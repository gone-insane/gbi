%% LaTeX-Beamer template for KIT design
%% by Erik Burger, Christian Hammer
%% title picture by Klaus Krogmann
%%
%% version 2.1
%%
%% mostly compatible to KIT corporate design v2.0
%% http://intranet.kit.edu/gestaltungsrichtlinien.php
%%
%% Problems, bugs and comments to
%% burger@kit.edu

\documentclass[18pt]{beamer}

%% SLIDE FORMAT

% use 'beamerthemekit' for standard 4:3 ratio
% for widescreen slides (16:9), use 'beamerthemekitwide'

\usepackage{templates/beamerthemekit}
% \usepackage{templates/beamerthemekitwide}

%% TITLE PICTURE

% if a custom picture is to be used on the title page, copy it into the 'logos'
% directory, in the line below, replace 'mypicture' with the 
% filename (without extension) and uncomment the following line
% (picture proportions: 63 : 20 for standard, 169 : 40 for wide
% *.eps format if you use latex+dvips+ps2pdf, 
% *.jpg/*.png/*.pdf if you use pdflatex)

\titleimage{banner}

%% TITLE LOGO

% for a custom logo on the front page, copy your file into the 'logos'
% directory, insert the filename in the line below and uncomment it

\titlelogo{logo_150x150}

% (*.eps format if you use latex+dvips+ps2pdf,
% *.jpg/*.png/*.pdf if you use pdflatex)

%% TikZ INTEGRATION

% use these packages for PCM symbols and UML classes
% \usepackage{templates/tikzkit}
% \usepackage{templates/tikzuml}

% the presentation starts here

\title[Einf\"uhrung]{Tutorium 1:\\ Einf\"uhrung}
\subtitle{Something for XYZ 2009}
\author{Dominik Muth - dominik.muth@student.kit.edu}

\institute{Institut f\"ur Informatik}

% Bibliography

\usepackage[citestyle=authoryear,bibstyle=numeric,hyperref,backend=biber]{biblatex}
\addbibresource{templates/example.bib}
\bibhang1em

\begin{document}

% change the following line to "ngerman" for German style date and logos
\selectlanguage{english}

%title page
\begin{frame}
\titlepage
\end{frame}

%table of contents
\begin{frame}{Outline/Gliederung}
\tableofcontents
\end{frame}

\section{\"Uber Mich}
\begin{frame}{\"Uber Mich}
\begin{itemize}
\item Name: \cite{Dominik Muth} %\language
\item Studiengang: \cite{Informatik} %\language
\item E-Mail: \cite{dominik.muth@student.kit.edu} %\language
\end{itemize}
\end{frame}

\section{GBI, was ist das?}
\begin{frame}{GBI}
\begin{itemize}
\item Logik
\pause
\item Sprachen/Grammatiken
\pause
\item Relationen/Abbildungen
\pause
\item Graphen
\pause
\item Laufzeitabsch\"atzung
\pause
\item Automaten
\pause
\item Turingmaschinen
\item \dots
\end{itemize}
\end{frame}


\section{Organisatorisches}
\subsection{Allgemeines}
\begin{frame}{Termine}
\begin{itemize}
\item Vorlesung: Mi. 11:30 Uhr im Audimax
\item \"Ubung: Fr. 9:45 Uhr im Audimax
\item Klausur: in der Regel anfang M\"arz
\end{itemize}
\end{frame}


\begin{frame}{Links}
\begin{block}{Vorlesung}
\begin{itemize}
\item Website: http://gbi.ira.uka.de
\item Dozentin: tanja.schultz@kit.edu
\end{itemize}
\end{block}

\begin{block}{Fachschaft}
\begin{itemize}
\item Website: http://www.fsmi.uni-karlsruhe.de/
\item Forum: http://www.fsmi.uni-karlsruhe.de/forum/
\end{itemize}
\end{block}
\end{frame}


\subsection{\"Ubungsbl\"atter}
\begin{frame}{\"Ubungsbl\"atter}
\begin{itemize}
\item Abgabe: Freitag? ?:? Uhr im UG des Infobaus (gegen\"uber der Toiletten)
\item 50\% der Punkte zum bestehen n\"otig
\end{itemize}

\begin{exampleblock}{must have:}
\begin{itemize}
\item Handgeschrieben
\item Deckblatt
\item keine Plagiate
\end{itemize}
\end{exampleblock}
\end{frame}


\section{Aussagenlogik}
\begin{frame}{Aussagenlogik\\ - Logische Aussagen 1}
Einfache Logische Aussagen:
\begin{itemize}
\item Negation $\neg$A: "nicht A"
\pause
\item Logisches Und (A $\land$ B): "A und B"
\pause
\item Logisches Oder (A $\lor$ B): "A oder B"
\end{itemize}
\end{frame}

\begin{frame}{Aufgabenteil 1}
\begin{block}{Aufgabe 1}
Stelle die Wahrheitstabelle auf:\\
$(\neg A \land B) \lor \neg B$
\end{block}
\begin{block}{Aufgabe 2}
Stelle die Wahrheitstabelle auf:\\
$(A \land B \land \neg C) \lor (C \land A) \lor (C \land B)$
\end{block}
\end{frame}


\begin{frame}{Aussagenlogik\\ - Logische Aussagen 2}
\begin{itemize}
\item Implikation (A $\Rightarrow$ B): "Wenn A, dann B"
\pause
\item \"Aquivalenz (A $\Leftrightarrow$ B): "A genau dann, wenn B" (Implikation in beide Richtungen)
\end{itemize}
\end{frame}


\begin{frame}{Aufgabenteil 2}
\begin{block}{Aufgabe 1}
Stelle die Wahrheitstabelle auf:\\
$(A \Rightarrow B) \Rightarrow C$
\end{block}
\begin{block}{Aufgabe 2}
Sind die Beiden Aussagen \"Aquivalent?:\\
$\neg (A \Rightarrow B) \Leftrightarrow \neg (\neg A \lor B)$
\end{block}
\end{frame}


\section{Relationen}
\begin{frame}{Relationen \\ - Allgemein}

\begin{block} {Definition Kartesisches Produkt}
Das Kartesisches Produkt $A \times B$ enth\"allt alle Kombinationen (a,b) mit $a \in A$ und $b \in B$.
\end{block}

\begin{block} {Definition Relation}
$R \subseteq A \times B$:\\
Eine Relation ist die Teilmenge eines Kartesischen Produktes.\\
Andere Schreibweise: $xRy$, mit $(x, y) \in R$
\end{block}
\end{frame}


\begin{frame}{Besonderheiten}
\begin{block}{Linkstotal}
Eine Relation ist linkstotal wenn gilt:\\
	f\"ur jedes $a$ existiert \textit{mindestens} ein $b$ f\"ur welches gilt $(a, b) \in R$
\end{block}

\begin{block}{Rechtseindeutig}
Eine Relation ist rechtseindeutig wenn gilt:\\
	f\"ur kein $a$ existiert mehr als ein $b$ mit $(a, b) \in R$
\end{block}

\begin{block}{}
Eine Relation, welche sowohl linkstotal als auch rechtseindeutig ist, nennt man auch Abbildung oder Funktion
\end{block}
\end{frame}


\begin{frame}{Besonderheiten}
\begin{block}{Rechtstotal}
Eine Relation ist rechtstotal wenn gilt:\\
	f\"ur jedes $b$ existiert \textit{mindestens} ein $a$ f\"ur welches gilt $(a, b) \in R$\\
	rechtstotal $=$ surjektiv
\end{block}

\begin{block}{Linkseindeutig}
Eine Relation ist linkseindeutig wenn gilt:\\
	f\"ur kein $b$ existiert mehr als ein $a$ mit $(a, b) \in R$\\
	linkseindeutig $=$ injektiv
\end{block}

\begin{block}{}
Eine surjektive und injektive Relation nennt man bijektiv
\end{block}
\end{frame}


\begin{frame}{Aufgabenteil 3}
Vervollst\"andige folgende Tabelle, wobei gilt:
$x \in \mathbb{R}$ und $f(x) \in \mathbb{R}$\\

\begin{block}{}
\begin{tabular}{llll}
	\hline
	\textbf{rechtstotal} & \textbf{linkseindeutig} & \textbf{Begriff} & \textbf{Abbildung} \\
	\hline
	? & ? & ? & $f(x) = x^3-x$\\
	? & ? & ? & $f(x) = x^2$\\
	? & ? & ? & $f(x) = x^5$\\
	? & ? & ? & $f(x) = e^{x}$\\
	\hline
\end{tabular}
\end{block}

\pause
Wie ver\"andert sich die Tabelle, wenn $x \in \mathbb{N}$ und $f(x) \in \mathbb{N}$?
\end{frame}

\begin{frame}
	Ende - Muss ich noch machen!!
\end{frame}

\end{document}
