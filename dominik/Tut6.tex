\include{header}

\title[Tutorium 6]{GBI Tutorium Nr. $2^5$}
\subtitle{Tutorium 6}
\date{28. November 2012}

% Bibliography



\begin{document}

	%title page
	\begin{frame}
		\titlepage
	\end{frame}

	%table of contents
	\begin{frame}{Outline/Gliederung}
		\tableofcontents
	\end{frame}
	
	
	\section{\"Ubungsblatt 4}
	\begin{frame} {Übungsblatt 4}
		\begin{center}
			\Huge ?
		\end{center}
	\end{frame}	
	
	
	\section{\"Ubungsblatt 5}
	\begin{frame} {Übungsblatt 5}
		\begin{center}
			\Huge ?
		\end{center}
	\end{frame}	
		
	
	
	
	\section{Wiederholung} 
	\begin{frame} {Wiederholung - Quiz}
		\begin{itemize}
			\item Schleifeninvarianten sind immer eindeutig. 
			\only<2-> {\color{red}$X$}\\
			\color{black}
					
			\item Aus einer Schleifeninvariante lässt sich der Sinn des Algorithmus herleiten.
			\only<3-> {\color{red}$X$}\\
			\color{black}
	
			\item $\neg (A \land B) \Leftrightarrow \neg A \lor \neg B$
			\only<4-> {\color{darkgreen}$\surd$}\\
			\color{black}
			
			\item $x^2$ ist eine surjektive Abbildung
			\only<5-> {\color{darkgreen}$\surd$\color{red}$X$}\\
			\color{black}
		\end{itemize}
	\end{frame}
	
	
	
	\begin{frame} {Wiederholung - Aufgaben}
		\begin{block}{Relationen}
			Es sei A die Menge aller Kinobesucher in einer Vorstellung 
			und B die Menge aller Sitzplätze. 
			Die Abbildung f ordnet den Kinobesuchern die Sitzplätze zu:\\
			\hspace{50pt}	$ f : A \rightarrow B $
			\begin{itemize}
				\pause
				\item Was wünschen sich die Kinobesucher: 
				Eine injektive, surjektive oder bijektive Abbildung 
				auf die Sitzplätze? Was wünscht sich der Kinobesitzer?
				
				\pause
				\item Erklären Sie jeweils, was es im Kino bedeutet, wenn $f$ linkstotal, linkseindeutig, rechtstotal, rechtseindeutig ist.
				
				\pause
				\item In dieser Teilaufgabe nehmen wir an, 6 Kinobesucher 
				besuchten ein Kino mit 8 Plätzen. 
				Zeichnen Sie eine injektive Abbildung $f$. 
				Wie viele injektive Abbildungen gibt es?
			\end{itemize}
		\end{block}
	\end{frame}
	
	
	
	\begin{frame} {Wiederholung}
		\begin{block}{Schleifeninvarianz}
			Gegeben sei folgender Algorithmus:\\
			\begin{algorithmic}
				\State $x \gets a;$
				\State $y \gets b;$
				\State $p \gets 0;$
				\While{$x > 0$}
					\State $p \gets p + y$
					\State $x \gets x - 1$
				\EndWhile
			\end{algorithmic}
			
			\begin{itemize}
				\pause
				\item Was macht dieser Algorithmus?
				
				\pause
				\item Stellen Sie eine Schleifeninvariante über alle 
				Variablen auf
				
				\pause
				\item Beweisen Sie Ihre Schleifeninvariante
			\end{itemize}						
		\end{block}
	\end{frame}
	
	
	\section{Zahlensysteme}
	\begin{frame}{Zahlensysteme}
		\begin{block}{Definition}
			Eine Zahl wird mit $Num_x$ zur Basis x dargestellt.\\
			
			\pause
			Beispiel: $Num_{10}$ ist definiert durch:\\
			
			
			\begin{align}
			Num_{10}(\epsilon) &=& 0 \\
			\forall v \in Z_{10}^*\forall w \in Z_{10} : Num_{10}(vw) 
			&=& 10 \cdot Num_{10}(v) + Num_{10}(w)
			\end{align}
			
		\end{block}
		
		\begin{Lemma}
			$Num_{10}$ ist durch Gleichung $1$ und $2$ wohldefiniert.\\
			\pause
			Wie beweisen wir das?
		\end{Lemma}
	\end{frame}
	
	
	\begin{frame}{Allgemein}
		\begin{block}{Für $Num_x$}
			\begin{align}
			Num_{x}(\epsilon) &=& 0 \\
			\forall v \in Z_{x}^*\forall w \in Z_{x} : Num_{10}(vw) 
			&=& x \cdot Num_{x}(v) + Num_{x}(w)
			\end{align}
		\end{block}
	\end{frame}
	
		
	
	\begin{frame}{Beispiel}
		\begin{block}{Beispiel für $Num_3$}
			$Num_3(12012)$\\
		\pause
			Nach Gleichung 2 $\Rightarrow Num_3(12012) = 3*Num_3(1201) + Num_3(2)$\\
		\pause
			$\Rightarrow^* 3*\bigg(3*\Big(3*\big(3*Num_3(1) )+Num_3(2)\big)+ Num_2(0)\Big)+Num_3(1)\bigg) + Num_3(2) $\\
		\pause
			\vspace{5pt}
			$= 81* Num_3(1) + 27* Num_3(2) + 9* Num_3(0) + 3* Num_3(1) + Num_3(2)$\\
		\pause
			\vspace{5pt}
			$ = 81* 1 + 27* 2 + 9 * 0 + 3 * 1 + 2$\\
		\pause
			\vspace{5pt}
			$ = 140$
		\end{block}
		
		\pause
		Kann man das ganze noch in einer anderen Form darstellen?
	\end{frame}
	
	
	\begin{frame}{Aufgaben}
		\begin{itemize}
			\item $\forall m \in \mathbb{N}_0 : Num_4({3}^m) = ?$\\
				\visible<2->{
					\color{darkgreen} 
					$4^m - 1$ 
					\color{black}}
					\pause
			\item Schreibe einen Algorithmus um die $Num_5(w)$ zu berechnen, 
			mit $w \in Z_5^*$. \\
			w(i) gibt das Zeichen an der i-ten Stelle zurück.
		\end{itemize}
	\end{frame}
	
	
	
	\section{Übersetzungen}
	\begin{frame}{Übersetzungen}
	
	\end{frame}
	
	
	
	
	\section{Huffman-Codierung}
	\begin{frame}{Huffman-Codierung}
	
	\end{frame}		
	
	
	
	
	\section{Fragen}
	\begin{frame} {Fragen}
		\begin{itemize}
			\item Fragen zum Stoff?
			\item Fragen zum n\"achsten \"Ubungsblatt?
			\item Generelle Fragen?
			\item Feedback?
		\end{itemize}
	\end{frame}

		
	\begin{frame} {EOF}
		\begin{center}
			\includegraphics[scale=0.5]{graphics/eof5.png}\\
			\tiny $source: http://imgs.xkcd.com/comics/home\_organization.png$
		\end{center}
	\end{frame}

\end{document}
