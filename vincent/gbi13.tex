\include{header}
\subtitle{Foliensatz 13}
\date{31. Januar 2013}

\newcommand{\sq}{$\square$}
\newcommand{\da}{$\downarrow$}

\begin{document}

\begin{frame}
    \titlepage
\end{frame}

\begin{frame}{Outline/Gliederung}
    \tableofcontents
\end{frame}

\section{Wiederholung}
\begin{frame}{Wiederholung}
    \begin{itemize}
        \item Irgendwas zu Turingmaschinen
        \item Irgendwas zu Codierungen
        \item Irgendwas zu Relationen
        \item Reflexiv
        \item Transitiv
        \item Symmetrisch
    \end{itemize}
\end{frame}
\section{Unentscheidbare Probleme}
\begin{frame}{Unentscheidbare Probleme}
    Es gibt Probleme, die lassen sich mit einer Turing-Maschine (oder äquivalent: einem Java-Programm) nicht lösen. (Auch nicht mit unendlich viel Zeit und Platz.)\\
    Ein solches Problem ist nicht \emph{entscheidbar}
    \begin{block}{Entscheidbarkeit}
        Für ein entscheidbares Problem gibt es eine Turingmaschine, die für jede Eingabe hält und das Eingabewort entweder akzeptiert oder nicht.
    \end{block}
\end{frame}
\begin{frame}{Codierung von Turingmaschinen}
   Bisher haben wir eine Turingmaschine formal so geschrieben $T = \left( Z, Z_0, X, f, g, m \right)$. Wir bauen uns eine Codierung, die die ganze Turingmaschine in ein Wort $w_1$ ``packt''. \\
   Dieses Wort $w_1$ übergeben wir dann einer universellen Turingmaschine $U$, die 
   \begin{itemize}
       \item übeprüft, ob $w_1$ eine Turingmaschine $T$ codiert
       \item dann die Turingmaschine $T$ ``simuliert'' und als Eingabe $w_2$ verwendet
       \item und schließlich das Ergebnis davon ausgibt
   \end{itemize}
   \begin{figure}[h]
       \centering
       \includegraphics[scale=.5]{graphics/13/33928220.jpg}
   \end{figure}
\end{frame}
\begin{frame}{Halteproblem}
    \begin{block}{Satz}
        Es ist nicht möglich, eine Turingmaschine $U$ zu bauen, die für jede Turingmaschine $T$ (codiert als $w_1$) und jede Eingabe $w_2$  entscheidet, ob $T$ bei der Eingabe von $w_2$ hält.
    \end{block}
    Das lässt sich auch beweisen.
\end{frame}<++>
\section{Äquivalenzrelationen}
\end{document}
