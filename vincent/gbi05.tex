\include{header}
\subtitle{Foliensatz 5}
\date{15. November 2012}

\begin{document}

\begin{frame}
    \titlepage
\end{frame}

\begin{frame}{Outline/Gliederung}
    \tableofcontents
\end{frame}

\section{Kontextfreie Grammatiken}
\begin{frame}{Definition}
    \begin{block}{Definition}
        Die Menge $G = G \left( N, T, S, P \right)$ nennen wir \textbf{Kontextfreie Grammatik}.
    \end{block}
    \pause
    \invisible<1>{
        \begin{block}{Was ist was?}
            \begin{itemize}
                \item $N$: \textbf{Nichtterminalsymbol}
                \item $T$: \textbf{Terminalsymbol}
                \item $S$: \textbf{Startsymbol}
                \item $P$: \textbf{Produktionsmenge}
            \end{itemize}
    \end{block}}
\end{frame}

\begin{frame}{Ableitung}
    \begin{block}{Definition}
        Als Ableitung wird in der theoretischen Informatik der Vorgang bezeichnet, ein Wort nach den Regeln einer formalen Grammatik zu erzeugen.
    \end{block}
    Wir schreiben: $w \Rightarrow^i v$, wenn von der Ableitung von $v$ aus $w$ $i$ Ableitungsschritte liegen ($i \in \mathbb{N}$).
    \pause
    \invisible<1>{
    \begin{alertblock}{Vorsicht}
        \begin{align*}
            \Rightarrow \neq \rightarrow
        \end{align*}
        \begin{itemize}
            \item $\Rightarrow$ ist die Relation der Ableitung
            \item $\rightarrow$ ist die Relation der Produktion ($\in P$)
        \end{itemize}
    \end{alertblock}}
\end{frame}

\begin{frame}{Ableitung}
    \begin{block}{Frage}
        Was stimmt? Es ist $w_1, w_2 \in N \cup P$.
        \begin{itemize}
            \item $w_1 \rightarrow w_2$, daraus folt $w_1 \Rightarrow w_2$
            \item $w_1 \Rightarrow w_2$, daraus folgt $w_1 \rightarrow w_2$
        \end{itemize}
    \end{block}
\end{frame}

\begin{frame}{Sprache der kontextfreien Grammatik}
    \begin{block}{Definition}
        Sei $G$ eine kontextfreie Grammatik. Dann bezeichnen wir die Sprache $L = L\left(G\right)$ mit 
        \begin{align*}
            L = \left\{ w\in T^* \big| S \Rightarrow^* w\right\}
        \end{align*}
    \end{block}
    \pause
    \invisible<1>{\begin{block}{Was ist $\Rightarrow^*$?}
        Mit $\Rightarrow^*$ ist die \emph{reflexiv-transitive Hülle} der Ableitungsrelation gemeint.
    \end{block}}
\end{frame}

\begin{frame}{Musikgrammatik\\(by Nils Braun und Philipp Basler)}
    Gegeben ist die Grammatik
    \begin{align*}
        G = \left( \left\{ X\right\} ,\left\{ A, B, C, D\right\} , X , \left\{ X \rightarrow \epsilon \big| AX \big| BX \big| CX \big| DX \right\}\right)A
    \end{align*}
    \pause
    \invisible<1-1>{Leite $A$ ab! }\pause\invisible<1-2>{Leite $ABC$ ab! Finde weitere Wörter.}\\\pause
    \begin{figure}
        \invisible<1-3>{\includegraphics[height=30mm]{graphics/05/abba.png}}\pause\hfill
        \invisible<1-4>{\includegraphics[height=30mm]{graphics/05/acdc.jpg}}\pause\hfill
        \invisible<1-5>{\includegraphics[height=30mm]{graphics/05/adac.png}}\pause
    \end{figure}
    \invisible<1-6>{$L\left( G\right) = \left\{ A, B, C, D\right\}^*$, oder etwa nicht?}
\end{frame}

\section{Relationen}
\begin{frame}{Eigenschaften von Relationen}
    \begin{block}{Definition}
        Sei $R \subset A \times A$ eine (binäre) Relation auf der Menge $A$. Wir nennen $R$
        \begin{itemize}
            \item \textbf{reflexiv} falls gilt:
                \begin{align*}
                    \forall x \in A: \left( x,x\right) \in R
                \end{align*}
            \item\textbf{transitiv} falls gilt:
                \begin{align*}
                    \forall\, x,\, y,\, z \in A:\,\left( x,\, y\right) \in R \wedge \left(y,\, z\right) \in R \Rightarrow \left( x,\, z\right) \in R
                \end{align*}
            \item \textbf{symmetrisch} falls gilt:
                \begin{align*}
                    \forall\, x,\, y \in A:\,\left( x,\, y\right) \in R \Rightarrow \left( y,\, x\right) \in R
                \end{align*}
        \end{itemize}
    \end{block}
\end{frame}
\begin{frame}{Verknüpfung}
    \begin{block}{Definition}
        Zwei Relationen $R\subseteq M \times N ,\, S\subseteq N\times L$ definieren die Relation des \emph{Produktes von $R$ und $S$} als
        \begin{align*}
            R \circ S = \left\{\left( x,z\right)\in M\times L | \exists\, y \in N :\left( x,\, y\right) \in R \text{ und } \left( y, z\right) \in S\right\}
        \end{align*}
    \end{block}
    \pause
    \invisible<1>{
    \begin{block}{Potenzschreibweise}
        \begin{align*}
            R^0 = I_\mathrm{M} = \left\{ \left( x,x\right) | x\in M\right\} \text{ und }
            R^\mathrm{i + 1} = R^\mathrm{i} \circ R
        \end{align*}
    \end{block}
    \begin{block}{Reflexiv-Transitive-Hülle}
        \begin{align*}
            R^* = \bigcup_\mathrm{i = 0}^\infty R^\mathrm{i}
        \end{align*}
    \end{block}}
\end{frame}


\end{document}
