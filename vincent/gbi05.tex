%% LaTeX-Beamer template for KIT design
%% by Erik Burger, Christian Hammer
%% title picture by Klaus Krogmann
%%
%% version 2.1
%%
%% mostly compatible to KIT corporate design v2.0
%% http://intranet.kit.edu/gestaltungsrichtlinien.php
%%
%% Problems, bugs and comments to
%% burger@kit.edu

\documentclass[18pt]{beamer}
\usepackage[utf8x]{inputenc}
\usepackage{units}
\usepackage{booktabs}
\usepackage{amsmath}
\usepackage{algpseudocode}
\usepackage{enumitem}

%% Definitions
\DeclareMathOperator{\div2}{div}
\renewcommand{\algorithmicrequire}{\textbf{Input:}}
\renewcommand{\algorithmicensure}{\textbf{Output:}}
\algnewcommand\algorithmicto{\textbf{to}}
\algrenewtext{For}[3]{\algorithmicfor\ $#1 \gets #2$ \algorithmicto\ $#3$ \algorithmicdo}
\algnewcommand\algorithmicod{\textbf{od}}
\algrenewtext{EndWhile}{\algorithmicod}
\algrenewtext{EndFor}{\algorithmicod}
\AtBeginSection[]{%
\begin{frame}<beamer> % do nothing in handouts
    \frametitle{Überblick}
    \tableofcontents[sectionstyle=show/shaded,
    subsectionstyle=show/show/hide]
\end{frame}
}
\AtBeginSubsection[]{%
\begin{frame}<beamer> % do nothing in handouts
    \frametitle{Überblick}
    \tableofcontents[sectionstyle=show/shaded,
    subsectionstyle=show/shaded/hide]
\end{frame}
}


%% SLIDE FORMAT

% use 'beamerthemekit' for standard 4:3 ratio
% for widescreen slides (16:9), use 'beamerthemekitwide'

\usepackage{templates/beamerthemekit}
%\usepackage{templates/beamerthemekitwide}

%% TITLE PICTURE

% if a custom picture is to be used on the title page, copy it into the 'logos'
% directory, in the line below, replace 'mypicture' with the 
% filename (without extension) and uncomment the following line
% (picture proportions: 63 : 20 for standard, 169 : 40 for wide
% *.eps format if you use latex+dvips+ps2pdf, 
% *.jpg/*.png/*.pdf if you use pdflatex)

%\titleimage{mypicture}

%% TITLE LOGO

% for a custom logo on the front page, copy your file into the 'logos'
% directory, insert the filename in the line below and uncomment it

%\titlelogo{mylogo}
\titlelogo{empty_logo}

% (*.eps format if you use latex+dvips+ps2pdf,
% *.jpg/*.png/*.pdf if you use pdflatex)

%% TikZ INTEGRATION

% use these packages for PCM symbols and UML classes
% \usepackage{templates/tikzkit}
% \usepackage{templates/tikzuml}

% the presentation starts here

\title[GBI Tutorium]{GBI Tutorium Nr. }
\subtitle{Foliensatz 0333}
\date{6. November 2012}
\author{Vincent Hahn -- vincent.hahn@student.kit.edu}

\institute{}

% Bibliography

%\usepackage[citestyle=authoryear,bibstyle=numeric,hyperref,backend=biber]{biblatex}
%\addbibresource{templates/example.bib}
%\bibhang1em

\begin{document}

% change the following line to "ngerman" for German style date and logos, english: english
\selectlanguage{ngerman}

\begin{frame}
    \titlepage
\end{frame}

\begin{frame}{Outline/Gliederung}
    \tableofcontents
\end{frame}

\section{Relationen}
\begin{frame}{Eigenschaften von Relationen}
    \begin{block}{Definition}
        Sei $R \subset A \times A$ eine (binäre) Relation auf der Menge $A$. Wir nennen $R$
        \begin{itemize}
            \item \textbf{reflexiv} falls gilt:
                \begin{align*}
                    \forall x \in A: \left( x,x\right) \in R
                \end{align*}
            \item\textbf{transitiv} falls gilt:
                \begin{align*}
                    \forall\, x,\, y,\, z \in A:\,\left( x,\, y\right) \in R \wedge \left(y,\, z\right) \in R \Rightarrow \left( x,\, z\right) \in R
                \end{align*}
            \item \textbf{symmetrisch} falls gilt:
                \begin{align*}
                    \forall\, x,\, y \in A:\,\left( x,\, y\right) \in R \Rightarrow \left( y,\, x\right) \in R
                \end{align*}
        \end{itemize}
    \end{block}
\end{frame}
\begin{frame}{Verknüpfung}
    \begin{block}{Definition}
        Zwei Relationen $R\subseteq M \times N ,\, S\subseteq N\times L$ definieren die Relation des \emph{Produktes von $R$ und $S$} als
        \begin{align*}
            R \circ S = \left\{\left( x,z\right)\in M\times L | \exists\, y \in N :\left( x,\, y\right) \in R \text{ und } \left( y, z\right) \in S\right\}
        \end{align*}
    \end{block}
    \pause
    \invisible<1>{
    \begin{block}{Potenzschreibweise}
        \begin{align*}
            R^0 = I_\mathrm{M} = \left\{ \left( x,x\right) | x\in M\right\} \text{ und }
            R^\mathrm{i + 1} = R^\mathrm{i} \circ R
        \end{align*}
    \end{block}
    \begin{block}{Reflexiv-Transitive-Hülle}
        \begin{align*}
            R^* = \bigcup_\mathrm{i = 0}^\infty R^\mathrm{i}
        \end{align*}
    \end{block}}
\end{frame}


\end{document}
