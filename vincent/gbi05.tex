\include{header}
\subtitle{Foliensatz 5}
\date{22. November 2012}

\begin{document}

\begin{frame}
    \titlepage
\end{frame}

\begin{frame}{Outline/Gliederung}
    \tableofcontents
\end{frame}

\section{Übungsblatt 4}
\begin{frame}{Allgemeine Fehler, Fragen}
    \begin{block}{Allgemeines}
        \begin{itemize}
            \item Versucht formale Beweise und Aussagen zu schreiben.
            \item Vollständige Induktion: welches ist das erste Element?
            \item Schleifeninvarianten gelten ab dem Einstiegspunkt in die Schleife.
        \end{itemize}
    \end{block}
\end{frame}

\section{Wiederholung} 
\begin{frame} {Wiederholung - Quiz}
    \begin{itemize}
        \item $X=X$ ist eine Schleifeninvariante! 
            \only<2-> {\color{darkgreen}$\surd$}\\
            \color{black}
        \item $A \Rightarrow B \Leftrightarrow \neg A \lor B$
            \only<3-> {\color{darkgreen}$\surd$}\\
            \color{black}
    \end{itemize}
\end{frame}

\section{Kontextfreie Grammatiken}
\begin{frame}{Definition}
    \begin{block}{Definition}
        Die Menge $G = G \left( N, T, S, P \right)$ nennen wir \textbf{Kontextfreie Grammatik}.
    \end{block}
    \pause
    \invisible<1>{
        \begin{block}{Was ist was?}
            \begin{description}
                \invisible<1-1>{\item[N:] Menge von Nichtterminalsymbolen}\pause
                \invisible<1-2>{\item[T:] Menge von Terminalsymbolen}\pause
                \invisible<1-3>{\item[S:] $S \in N$ Startsymbol}\pause
                \invisible<1-4>{\item[$P \subset N \times V^*$:] Menge von Produktionen, \\
                $V := (N \cup T)$}
            \end{description}
        \end{block}}
\end{frame}

\begin{frame}{Ableitung}
    \begin{block}{Definition}
        Als Ableitung wird in der theoretischen Informatik der Vorgang bezeichnet, ein Wort nach den Regeln einer formalen Grammatik zu erzeugen.
    \end{block}
    Wir schreiben: \pause
    \begin{itemize}
        \item $w \Rightarrow v$, wenn von der Ableitung von $v$ aus $w$ genau $1$ Ableitungsschritt liegt.\pause
        \item $w \Rightarrow^i v$, wenn von der Ableitung von $v$ aus $w$ $i$ Ableitungsschritte liegen ($i \in \mathbb{N}$).\pause
        \item $w \Rightarrow^* v$, wenn von der Ableitung von $v$ aus $w$ beliebig viele Ableitungsschritte liegen.
    \end{itemize}
\end{frame}

\begin{frame}{Ableitung}
    \begin{alertblock}{Vorsicht}
        \begin{align*}
            \Rightarrow \neq \rightarrow
        \end{align*}
        \begin{itemize}
            \item $\Rightarrow$ ist die Relation der Ableitung
            \item $\rightarrow$ ist die Relation der Produktion ($\in P$)
        \end{itemize}
    \end{alertblock}
\end{frame}

\begin{frame}{Ableitung}
    \begin{block}{Frage}
        Was stimmt? Es ist $w_1, w_2 \in N \cup P$.
        \begin{itemize}
            \item $w_1 \rightarrow w_2$, daraus folgt $w_1 \Rightarrow w_2$ \invisible<1>{Wahr.}
            \item $w_1 \Rightarrow w_2$, daraus folgt $w_1 \rightarrow w_2$ \invisible<1>{Falsch.}
        \end{itemize}
        \pause
    \end{block}
\end{frame}

\begin{frame}{Ableitungsbäume}
    \begin{block}{Klammern}
        Gegeben ist die Sprache $G = \left( \left\{ X\right\} , \left\{ \left( ,\right) \right\} , X, \left\{ X \rightarrow XX \big| \left( X\right) \big| \varepsilon \right\} \right) $\\
        Zeichne einen Ableitungsbaum für die Ausdrücke
        \begin{itemize}
            \item (( ))( )
            \item (( ))( )((( ))) 
        \end{itemize}
        Welche Eigenschaften hat ein Wort $w$ dieser Grammatik?
        \begin{itemize}
            \item \only<2->{wohlgeformter Klammerausdruck}
            \item \only<2->{$N_( \left( w \right)=N_) \left( w \right)$}
            \item \only<2->{Für jedes Präfix $v$: $N_( \left( v \right)\geq N_) \left( v \right)$}
        \end{itemize}
    \end{block}
\end{frame}

\begin{frame}{Sprache der kontextfreien Grammatik}
    \begin{block}{Definition}
        Sei $G$ eine kontextfreie Grammatik. Dann bezeichnen wir die Sprache $L = L\left(G\right)$ mit 
        \begin{align*}
            L = \left\{ w\in T^* \big| S \Rightarrow^* w\right\}
        \end{align*}
    \end{block}
    \pause
    \begin{block}{Was ist $\Rightarrow^*$?}\pause
        \invisible<1-2>{Mit $\Rightarrow^*$ ist die \emph{reflexiv-transitive Hülle} der Ableitungsrelation gemeint.}
    \end{block}
\end{frame}

\begin{frame}
    \frametitle{Quickies}
    \begin{exampleblock}{Fragen}
        \begin{enumerate}
          \item Gibt es Grammatiken für die gilt: $L(G) = \{\}$?\pause
          \item Welche Sprache erzeugt: $G_1 := (\{X, S\}, \{0\}, S, \{X \rightarrow X, S\rightarrow X, X \rightarrow \epsilon\})$\pause
          \item Ist $G_2 := (\{S\}, \{a, b\}, S, \{S \rightarrow \varepsilon\})$ eine gültige Grammatik?\pause
        \end{enumerate}
    \end{exampleblock}
    \begin{exampleblock}{Lösung}
        \begin{enumerate}
            \invisible<1-1>{\item Ja z.B. $G = (\{S\}, \{0\}, S, \{ S\rightarrow \varepsilon\})$}
            \invisible<1-2>{\item $L(G_1) = \{\}$}
            \invisible<1-3>{\item Ja und $L(G_2) = \{\varepsilon\}$}
        \end{enumerate}
    \end{exampleblock}
\end{frame}

\begin{frame}{Musikgrammatik\\(by Nils Braun und Philipp Basler)}
    Gegeben ist die Grammatik
    \begin{align*}
        G = \left( \left\{ X\right\} ,\left\{ A, B, C, D\right\} , X , \left\{ X \rightarrow \epsilon \big| AX \big| BX \big| CX \big| DX \right\}\right)
    \end{align*}
    \pause
    \invisible<1-1>{Leite $A$ ab! }\pause\invisible<1-2>{Leite $ABC$ ab! Finde weitere Wörter.}\\\pause
    \begin{figure}
        \invisible<1-3>{\includegraphics[height=30mm]{graphics/05/abba.png}}\pause\hfill
        \invisible<1-4>{\includegraphics[height=30mm]{graphics/05/acdc.jpg}}\pause\hfill
        \invisible<1-5>{\includegraphics[height=30mm]{graphics/05/adac.png}}\pause
    \end{figure}
    \invisible<1-6>{$L\left( G\right) = \left\{ A, B, C, D\right\}^*$, oder etwa nicht?}
\end{frame}

\begin{frame}
  \frametitle{Aufgaben}
  \begin{exampleblock}{Welche Sprachen erzeugen folgende Grammatiken.}
    \begin{enumerate}
      \item $G_1 := (\{X, Y\}, \{a, b\}, X, \{X \rightarrow aY | \varepsilon, Y \rightarrow bX\})$
      \item $G_2 := (\{X, Y, Z\}, \{a, b, c\}, X,$\\
             $\{X \rightarrow Ya | Yb | Yc, Y \rightarrow ZZY | \varepsilon, Z \rightarrow a | b | c\})$
    \end{enumerate}
  \end{exampleblock}
  \begin{exampleblock}{Gebt eine jeweils Grammatik an für die gilt $L(G) = L_i$:}
        \begin{enumerate}
      \item $L_1 := \{ab, cd\}^+ \cdot \{a, c\}^2$
      \item $A := \{0, 1\}$, $L_2 := \{w  \in A^*| Num_0(w) = Num_1(w)\}$
    \end{enumerate}
  \end{exampleblock}
\end{frame}

\begin{frame}{Bonus-Aufgabe (by Patrick Niklaus)}
  \begin{exampleblock}{In Mengen M aus Studenten mit $|M| \leq 3$}
    Konstruiert eine Grammatik die alle E-Mail-Adresse aus den Buchstaben {a, b, c} erzeugt.
    \emph{Hinweis}: $T := \{a, b, c, @, ., \_\}$
  \end{exampleblock}\pause
  \invisible<1>{\begin{exampleblock}{Lösung}
      Achtung: die Lösung deckt nicht alle korrekten Fälle ab.
    $G = (N, T, S, P)$
    \begin{itemize}
      \item $N = \{E, A, B\}$
      \item $T = \{a, b, c, ., \_, @\}$
      \item $S = E$
      \item $P = \{E \longrightarrow A@B.B, A \longrightarrow BA|\_A|.A, B \longrightarrow aB | bB | cB | \varepsilon\}$
    \end{itemize}
    \end{exampleblock}}
\end{frame}

\begin{frame}{Aufgabe: Winter 2008/2009}
    \begin{exampleblock}{Aufgabe}
        \begin{itemize}
            \item Geben Sie eine kontextfreie Grammatik 
                \begin{align*}
                    G = \left( N,\left\{ a,b\right\} ,S,P\right) 
                \end{align*}
                an, für die $L\left( G\right) $ die Menge aller Palindrome über dem Alphabet $\left\{ a,b\right\} $ ist. 
            \item Geben Sie eine Ableitung der Wörter $baaab$ und $abaaaba$ aus dem Startsymbol Ihrer Grammatik an. 
            \item Beweisen Sie, dass Ihre Grammatik jedes Palindrom über dem Alphabet $\left\{ a, b\right\} $ erzeugt.
        \end{itemize}
    \end{exampleblock}
\end{frame}

\section{Relationen}
\begin{frame}{Relationen}
  \begin{definition}
    \begin{itemize}
      \item Seien A, B zwei Mengen. $ R \subseteq A \times B $\pause
      \item R ist eine Teilmenge des Kreuzproduktes zweier Mengen und heißt \emph{Relation}.\pause
      \item Man schreibt auch: xRy für $(x, y) \in R$\pause
      \item Ist A = B so nennt man R auch eine \emph{homogene} Relation.\pause
    \end{itemize}
  \end{definition}
\end{frame}

\begin{frame}{Eigenschaften von Relationen}
    \begin{block}{Definition}
        Sei $R \subset A \times A$ eine (binäre) Relation auf der Menge $A$. Wir nennen $R$
        \begin{itemize}
            \item \textbf{reflexiv} falls gilt:
                \begin{align*}
                    \forall x \in A: \left( x,x\right) \in R
                \end{align*}
            \item\textbf{transitiv} falls gilt:
                \begin{align*}
                    \forall\, x,\, y,\, z \in A:\,\left( x,\, y\right) \in R \wedge \left(y,\, z\right) \in R \Rightarrow \left( x,\, z\right) \in R
                \end{align*}
            \item \textbf{symmetrisch} falls gilt:
                \begin{align*}
                    \forall\, x,\, y \in A:\,\left( x,\, y\right) \in R \Rightarrow \left( y,\, x\right) \in R
                \end{align*}
        \end{itemize}
    \end{block}
\end{frame}

\begin{frame}{Aufgaben}
    \begin{block}{Quickies}
        \begin{itemize}
            \item Ist die Relation $\geq$ auf $\mathbb{N}_0$ reflexiv? \pause \invisible<1>{Ja.}
            \item Ist die Relation $>$ auf $\mathbb{N}_0$ reflexiv? \pause \invisible<1-2>{Nein.}
            \item Ist die Relation $\geq$ auf $\mathbb{N}_0$ transitiv? \pause \invisible<1-3>{Ja.}
            \item Ist die Relation $\geq$ auf $\mathbb{N}_0$ symmetrisch? \pause \invisible<1-4>{Nein.}
            \item Ist die Relation $=$ auf $\mathbb{N}_0$ symmetrisch? \pause \invisible<1-5>{Ja.}
        \end{itemize}
    \end{block}
\end{frame}

\begin{frame}{Aufgaben 2}
    \begin{exampleblock}{Quickies}
        \begin{itemize}
            \item Ist die Relation: 
                $
                R = f(x) =
                \begin{cases}
                x + 1 & \text{falls x gerade}\\
                x - 1 & \text{falls x ungerade}
                \end{cases}
                $
                \\ mit $x\in \mathbb{N}_0$ und $f(x) \in \mathbb{N}_0$ symmetrisch?
            \item Ist die Relation: $R = \{(x,y)\in M \times M \;|\; y = x^2\} $ reflexiv?
        \end{itemize}
    \end{exampleblock}
\end{frame}

\begin{frame}{Produkt}
    \begin{block}{Definition}
        Zwei Relationen $R\subseteq M \times N ,\, S\subseteq N\times L$ definieren die Relation des \emph{Produktes von $R$ und $S$} als
        \begin{align*}
            R \circ S = \left\{\left( x,z\right)\in M\times L | \exists\, y \in N :\left( x,\, y\right) \in R \text{ und } \left( y, z\right) \in S\right\}
        \end{align*}
    \end{block}
    \pause
    \invisible<1>{
    \begin{block}{Potenzschreibweise}
        \begin{align*}
            R^0 = I_\mathrm{M} = \left\{ \left( x,x\right) | x\in M\right\} \text{ und }
            R^\mathrm{i + 1} = R^\mathrm{i} \circ R
        \end{align*}
    \end{block}
    \begin{block}{Reflexiv-Transitive-Hülle}
        \begin{align*}
            R^* = \bigcup_\mathrm{i = 0}^\infty R^\mathrm{i}
        \end{align*}
    \end{block}}
\end{frame}

\begin{frame}{Äquivalenzrelationen}
	\begin{definition}
    Sei R eine homogene Relation über M. $x, y, z \in M$.\\
    Hat R folgende Eigenschaften:
		\begin{description}
			\item[reflexiv] $x R x$
			\item[transitiv] $x R y \wedge y R z \Rightarrow x R z$
			\item[symmetrisch] $x R y \Rightarrow y R x$
		\end{description}
		So heißt R eine \emph{Äquivalenzrelation}.
	\end{definition}
\end{frame}

\begin{frame}{Beispiel (by Patrick Niklaus)}
	\begin{exampleblock}{Freunde im Netzwerk}
    {\small
			Sei $M = \{ Gertrud, Holger, Lars, Katja, Martin, Nina \}$ eine Menge von Nutzern.\\
			$R \subseteq M \times M $ sei die "`ist-befreundet-mit"'-Relation.
		\begin{itemize}
			\item $R = \{ (Martin,Holger), (Lars,Katja), (Nina,Holger),$ \\
						$(Gertrud,Holger), (Katja, Nina) \} \bigcup \{${dazu sym. Tupel}$\}$ \pause
			\item $R^0=\{ (Martin,Martin), ..., (Holger,Holger) \}$ \pause
			\item $R^1=R$ "'Freundschaft 1. Grades."' \pause
			\item $R^2=\{ (Martin,Nina), (Martin,Gertrud), (Martin,Martin),$ \\
						$(Lars,Nina), (Lars,Lars), (Nina,Gertrud),(Nina,Martin),$ \\
						$(Nina,Nina), (Nina,Lars), (Katja,Katja), (Katja,Holger), $ \\
						$(Gertrud,Gertrud), (Gertrud,Martin), (Gertrud,Nina), $ \\
						$(Holger,Holger), (Holger,Katja)\}$ "'Freundschaft 2. Grades"' \pause
			\item $R^*=?$ "'Gibt es eine Verbindung durch Freunde beliebigen Grades?"'
			%\item wegen Symmetrie und Unsinnigkeit der Reflexivität, entfällt Einiges % Funktionen sind kommutativ, Relationen symmetrisch
		\end{itemize}
    }
	\end{exampleblock}
\end{frame}

\begin{frame}{Aufgaben}
  \begin{exampleblock}{Reflexiv-transitive Hülle}
    Bestimmt die reflexiv-transitive Hülle der Relationen.\\
    $M := \{1, 2, 3, 4\}, R_i \subset M \times M$
    \begin{enumerate}
      \item $R_1 := \{(1, 2), (1, 4), (2, 3)\}$
      \item $R_2 := \{(1, 1), (2, 2)\}$
      \item $R_3 := \{(1, 2), (3, 4), (4, 2)\}$
    \end{enumerate}
  \end{exampleblock}\pause
  \invisible<1>{\begin{exampleblock}{Lösungen}
    \begin{enumerate}
      \item $R_1^* := \{(1, 1), (1, 2), (1, 3), (1, 4), (2, 2), (2, 3), (3, 3), (4, 4)\}$
      \item $R_2^* := \{(1, 1), (2, 2), (3, 3), (4, 4)\}$
      \item $R_3^* := \{(1, 1), (1, 2), (2, 2), (3, 2), (3, 3), (3, 4), (4, 2), (4, 4)\}$
    \end{enumerate}
    \end{exampleblock}}
\end{frame}
\end{document}
