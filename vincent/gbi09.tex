%% LaTeX-Beamer template for KIT design
%% by Erik Burger, Christian Hammer
%% title picture by Klaus Krogmann
%%
%% version 2.1
%%
%% mostly compatible to KIT corporate design v2.0
%% http://intranet.kit.edu/gestaltungsrichtlinien.php
%%
%% Problems, bugs and comments to
%% burger@kit.edu

\documentclass[18pt]{beamer}
\usepackage[utf8x]{inputenc}
\usepackage{units}
\usepackage{booktabs}

%% CUSTOM
\usepackage{amsmath}
\usepackage{algpseudocode}

%% Definitions
\DeclareMathOperator{\div2}{div}
\renewcommand{\algorithmicrequire}{\textbf{Input:}}
\renewcommand{\algorithmicensure}{\textbf{Output:}}
\algnewcommand\algorithmicto{\textbf{to}}
\algrenewtext{For}[3]{\algorithmicfor\ $#1 \gets #2$ \algorithmicto\ $#3$ \algorithmicdo}
\algnewcommand\algorithmicod{\textbf{od}}
\algrenewtext{EndWhile}{\algorithmicod}
\algrenewtext{EndFor}{\algorithmicod}
%\AtBeginSection[]{%
%\begin{frame}<beamer> % do nothing in handouts
%    \frametitle{Überblick}
%    \tableofcontents[sectionstyle=show/shaded,
%    subsectionstyle=show/show/hide]
%\end{frame}
%}
%\AtBeginSubsection[]{%
%\begin{frame}<beamer> % do nothing in handouts
%    \frametitle{Überblick}
%    \tableofcontents[sectionstyle=show/shaded,
%    subsectionstyle=show/shaded/hide]
%\end{frame}
%}

%% SLIDE FORMAT

% use 'beamerthemekit' for standard 4:3 ratio
% for widescreen slides (16:9), use 'beamerthemekitwide'

\usepackage{templates/beamerthemekit}
%\usepackage{templates/beamerthemekitwide}

 %% TITLE PICTURE

 % if a custom picture is to be used on the title page, copy it into the 'logos'
 % directory, in the line below, replace 'mypicture' with the 
 % filename (without extension) and uncomment the following line
 % (picture proportions: 63 : 20 for standard, 169 : 40 for wide
 % *.eps format if you use latex+dvips+ps2pdf, 
 % *.jpg/*.png/*.pdf if you use pdflatex)


 \titleimage{banner}
 
 
%% Define some colors:
\definecolor{darkblue}{rgb}{0,0,.5}
\definecolor{darkgreen}{rgb}{0,.5,0}

 %% TITLE LOGO

 % for a custom logo on the front page, copy your file into the 'logos'
 % directory, insert the filename in the line below and uncomment it

\titlelogo{logo_150x150}
 
 % (*.eps format if you use latex+dvips+ps2pdf,
 % *.jpg/*.png/*.pdf if you use pdflatex)
 
 %% TikZ INTEGRATION
 
 % use these packages for PCM symbols and UML classes
 % \usepackage{templates/tikzkit}
 % \usepackage{templates/tikzuml}
 
 % the presentation starts here
 
\author{Dominik Muth - dominik.muth@student.kit.edu}
\institute{Institut f\"ur Informatik}

\subtitle{Foliensatz 8}
\date{13. Dezember 2012}

\begin{document}

\begin{frame}
    \titlepage
\end{frame}

\begin{frame}{Outline/Gliederung}
    \tableofcontents
\end{frame}

\section{O-Kalkül/Landau-Symbole}
\begin{frame}{Asymptotisches Wachstum}
    \begin{block}{Definition}
        Zwei Funktionen $f,g: \mathbb{N}_0 \rightarrow \mathbb{R}_0^+$ wachsen asymptotische genauso schnell, wenn es zwei Konstanten $c, c' \in \mathbb{R}^+$ gibt, so dass gilt:
        \begin{align*}
            \exists n' \in\mathbb{N}_0 \forall n > n': c\cdot f\left( n \right) \leq g\left( n \right) \geq c'\cdot f\left( n \right)
        \end{align*}
        Wir schreiben dafür auch
        \begin{align*}
            f\asymp g
        \end{align*}
    \end{block}
\end{frame}
\begin{frame}{Asymptotisches Wachstum}
    Welche Eigenschaften hat diese Relation?
    \begin{align*}
        f \asymp g
    \end{align*}
    \visible<2->{\begin{itemize}
        \item symmetrisch
        \item reflexiv
        \item transitiv
    \end{itemize}
    Damit ist dies eine Äquivalenzrelation.}
\end{frame}
\begin{frame}{Äquivalenzklassen}
    \begin{block}{Definition}
        $\Theta\left( f \right)$ ist die \emph{Menge} aller Funktionen $g$, die asymptotisch genauso schnell wachsen wie $f$, also
        \begin{align*}
            \Theta\left( f \right) = \left\{ g | f\asymp g \right\}
        \end{align*}
    \end{block}
\end{frame}
\begin{frame}{Quickies}
    Ist $8\cdot x^2 \in \Theta\left( x^2 \right)$?
    \visible<2->{Ja, denn es gilt:
    \begin{align*}
        8\cdot x^2 \asymp x^2
    \end{align*}
    Ist $x^3 \in \Theta{x^2}$? Und gilt $e^x \in \Theta{x^2}$?\\}
    \visible<3->{Nein.}
    \visible<4->{Ist $x^3 + x^2 \in \Theta\left( x^3 \right)$?\\}
    \visible<5->{Ja!}
\end{frame}
\begin{frame}{Asympototisches Wachstum}
    \begin{block}{Definition}
        Für zwei Funktionen $f,g: \mathbb{N}_0\rightarrow \mathbb{R}_0^+$ definiert man:
        \begin{align*}
            g\preceq f & \exists c \in \mathbb{R}^+: \exists n' \in \mathbb{N}_0: \forall n > n': & g\left( n \right) \leq c\cdot f\left( n \right)\\
            g\color{red}{\succeq} f & \exists c \in \mathbb{R}^+: \exists n' \in \mathbb{N}_0: \forall n > n': & g\left( n \right) \color{red}{\geq} c\cdot f\left( n \right)
        \end{align*}
    \end{block}
    Umgangssprächlich: ``wächst asymptotisch höchstens so schnell wie'' oder ``wächst asymptotisch mindestens so schnell wie''.
\end{frame}
\begin{frame}{Äquivalenzklassen}
    \begin{block}{Definition}
        $O(f)$ ist die Menge aller Funktionen, die asymptotische höchstens so schnell wachsen wie f.
        \begin{align*}
            O\left( f \right) = \left\{ g\big| g\preceq f \right\}
        \end{align*}
    \end{block}
    \begin{block}{Definition}
        $\Omega(f)$ ist die Menge aller Funktionen, die asymptotische mindestens so schnell wachsen wie f.
        \begin{align*}
            \Omega\left( f \right) = \left\{ g\big| g\succeq f \right\}
        \end{align*}
    \end{block}
\end{frame}
\begin{frame}{Quickies}
   Was stimmt?
   \begin{itemize}
       \item $x^3 + x^2 \in \Omega\left( x^2 \right)$ \visible<2->{Wahr.}
       \item $x^3 + x^2 \in \Omega\left( x^3 \right)$ \visible<3->{Wahr.}
       \item $x^3 + x^2 \in O\left( x^2 \right)$ \visible<4->{Falsch.}
       \item $x^3 + x^2 \in O\left( x^4 \right)$ \visible<5->{Wahr.}
       \item $e^x \in O\left( x^4 \right)$ \visible<6->{Falsch.}
       \item $e^x \in \Omega\left( x^4 \right)$ \visible<7->{Wahr.}
   \end{itemize}
\end{frame}
\begin{frame}{Ungenauigkeiten}
    Häufig wird auch das geschrieben
    \begin{alertblock}{Achtung: Falsche Schreibweise}
        \begin{align*}
            f = \Theta\left( g \right) & h = O\left( n^3 \right) & k = \Omega\left( f + g \right)
        \end{align*}
        Das gibt Punkteabzug.
    \end{alertblock}
    Was genau ist hier falsch?\\
    \visbile<2->{Links vom Gleichheitszeichen steht jeweils eine Funktion, rechts eine Menge. Das ist wie ``Apfel ist gleich Korb'' - einfach falsch.}
\end{frame}
\begin{frame}{Formalitäten}
    Untersuchen von zwei Polynom-Funktionen:
    \begin{itemize}
        \item $f\left( n \right) = n^4 + n^3$
        \item $g\left( n \right) = n^2$
    \end{itemize}
    \hline
    \begin{align*}
        \frac{f\left( n \right)}{g\left( n \right)} = \frac{n^4+n^3}{n^2} = n^2 + n
    \end{align*}
    \hline
    \begin{itemize}
        \item Falls $\lim\limits_{n\rightarrow \infty}{\frac{f\left( n \right)}{g\left( n \right)}} = 1$. Dann: $f \in \Theta\left( g \right)$. In diesem Fall: Nicht
        \item Falls $\lim\limits_{n\rightarrow \infty}{\frac{f\left( n \right)}{g\left( n \right)}} = \infty$. Dann: $g \in O\left( f \right)$ und dann auch: $f \in \Omega\left( g \right)$. Das stimmt hier
    \end{itemize}
\end{frame}
\begin{frame}{Wer gewinnt?}
    \begin{table}
        \centering
        \begin{tabular}{ll}
            \toprule
            Notation & Beispiel\\
            \midrule
            $f \in \mathcal{O}\left( 1 \right)$ & $f = 5$\\
            $f \in \mathcal{O}\left( \log\left( n\right) \right)$ & $f = \ln\left( x \right)$\\
            $f \in \mathcal{O}\left( \sqrt{n} \right)$ & $f= 2\cdot \sqrt{n}$\\
            $f \in \mathcal{O}\left( n \right)$ & $f = 7\cdot n$\\
            $f \in \mathcal{O}\left( \log\left( n \right)\cdot n \right)$ & $f = 3\cdot n\cdot \ln\left( n \right)$\\
            $f \in \mathcal{O}\left( n^2 \right)$ & $f = 3\cdot n^2$\\
            $f \in \mathcal{O}\left( 2^n \right)$ & $f = 3\cdot 2^n$\\
            $f \in \mathcal{O}\left( n! \right)$ & $f = 3\cdot n!$\\
            \bottomrule
        \end{tabular}
    \end{table}
\end{frame}
\begin{frame}{Logarithmus-Beispiel}
    Zur Veranschaulichung:
    \begin{table}
        \centering
        \begin{tabular}{rrr}
            \toprule
            $n$ & $\log_8 n$ & $\log_2 n$\\
            \midrule
            1 & 0 & 0 \\
            8 & 1 & 3 \\
            64 & 2 & 6\\
            512 & 3 & 9 \\
            4096 & 4 & 12 \\
            \bottomrule
        \end{tabular}
    \end{table}
    Was fällt auf?\\
    \visible<2->{$\log_2 n = 3\cdot \log_8 n$. Die beiden Logarithmen unterscheiden sich also nur durch einen konstanten Faktor.}
\end{frame}
\begin{frame}{Logarithmus}
    Aus den Logarithmusregeln folgt:
    \begin{align}
        n &= a^{\log_a \left( n \right)}\label{eqn:1}\\
        n &= b^{\log_b\left( n \right)}\label{eqn:2}\\
        \visible<2->{\eqref{eqn:1} \Rightarrow (a)^{\log_a n} &= \left( b^{\log_b a} \right)^{\log_a n} = b^{\log_b a\cdot \log_a n}\label{eqn:3}\\
        \eqref{eqn:2} \text{ und } \eqref{eqn:3} \Rightarrow (b)^{\log_b n} &= b^{\log_b a\cdot \log_a n}\label{eqn:4}}
    \end{align}
    Was bringt das jetzt? Wir können zeigen, dass alle Logarithmen, egal zu welcher Basis, asymptotisch wachsen. Setze
    $c = c' = \log_b a$. Dann:
    \begin{align*}
        c \log_a n \leq \log_b n \leq c' \log_a n
    \end{align*}
\end{frame}
\begin{frame}{Rechenregeln}
    Das dürft ihr so verwenden:
    \begin{itemize}
        \item $f \in \mathcal{O}\left( g \right) \leftrightarrow g \in \Omega\left( f \right)$
        \item $\Theta\left( f  \right) = \mathcal{O}\left( f \right) \cap \Omega\left( f \right)$ oder\\
            $f \asymp g \leftrightarrow f \succeq g \wedge f \preceq g $
        \item $O\left( f_1 \right) + O\left( f_2 \right) = O\left( f_1 + f_2 \right)$
        \item Wenn $g \in \mathca{O}\left( f \right)$, dann ist auch $\mathcal{O}\left( g \right)\subseteq \mathcal{O}\left( f \right)$ und $O\left( f + g \right) = \mathcal{O}\left( f \right)$
    \end{itemize}
\end{frame}

\section{Laufzeiten}
\begin{frame}{Beispielalgorithmus}
    \begin{algorithm}
        \begin{algorithmic}
            \Require $a \in \mathbb{R}$
            \Require $n \in \mathbb{N}_+$
            \State $x \gets 1$
            \For{i}{1}{n}
                \State $x\gets a\cdot x$
            \EndFor
            \Ensure $x$
        \end{algorithmic}
    \end{algorithm}
    Was passiert und in welcher Laufzeit?
\end{frame}
\begin{frame}{Klassen}
    Welche Funktionen gehören in welche Klasse(n)?
    \begin{table}
        \centering
        \begin{tabular}{lcccc}
            \toprule
            & $\mathcal{O}\left( n^2 \right)$ & $\Omega\left( n^2 \right)$ & $\mathcal{O}\left( \log n \right)$ & $\Theta\left( n \right)$\\
             \midrule
             $n^2 + n$ & j & j & n & n\\
             $n\cdot \log n$ & j & n & n & n\\
             $2\cdot n + 1$ & j & n & n & j\\
             $n^3$ & n & j & n & n\\
             $5$ & j & n & j & n\\
             \bottomrule
        \end{tabular}
    \end{table}
\end{frame}

\end{document}
