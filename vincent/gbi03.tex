%% LaTeX-Beamer template for KIT design
%% by Erik Burger, Christian Hammer
%% title picture by Klaus Krogmann
%%
%% version 2.1
%%
%% mostly compatible to KIT corporate design v2.0
%% http://intranet.kit.edu/gestaltungsrichtlinien.php
%%
%% Problems, bugs and comments to
%% burger@kit.edu

\documentclass[18pt]{beamer}
\usepackage[utf8x]{inputenc}
\usepackage{units}
\usepackage{booktabs}


%% SLIDE FORMAT

% use 'beamerthemekit' for standard 4:3 ratio
% for widescreen slides (16:9), use 'beamerthemekitwide'

\usepackage{templates/beamerthemekit}
%\usepackage{templates/beamerthemekitwide}

%% TITLE PICTURE

% if a custom picture is to be used on the title page, copy it into the 'logos'
% directory, in the line below, replace 'mypicture' with the 
% filename (without extension) and uncomment the following line
% (picture proportions: 63 : 20 for standard, 169 : 40 for wide
% *.eps format if you use latex+dvips+ps2pdf, 
% *.jpg/*.png/*.pdf if you use pdflatex)

%\titleimage{mypicture}

%% TITLE LOGO

% for a custom logo on the front page, copy your file into the 'logos'
% directory, insert the filename in the line below and uncomment it

%\titlelogo{mylogo}
\titlelogo{empty_logo}

% (*.eps format if you use latex+dvips+ps2pdf,
% *.jpg/*.png/*.pdf if you use pdflatex)

%% TikZ INTEGRATION

% use these packages for PCM symbols and UML classes
% \usepackage{templates/tikzkit}
% \usepackage{templates/tikzuml}

% the presentation starts here

\title[GBI Tutorium]{GBI Tutorium Nr. }
\subtitle{Foliensatz 0333}
\date{6. November 2012}
\author{Vincent Hahn -- vincent.hahn@student.kit.edu}

\institute{Institut für theoretische Informatik}

% Bibliography

%\usepackage[citestyle=authoryear,bibstyle=numeric,hyperref,backend=biber]{biblatex}
%\addbibresource{templates/example.bib}
%\bibhang1em

\begin{document}

% change the following line to "ngerman" for German style date and logos, english: english
\selectlanguage{ngerman}

\begin{frame}
    \titlepage
\end{frame}

\begin{frame}{Outline/Gliederung}
    \tableofcontents
\end{frame}

\section{Formale Sprachen}
\begin{frame}{Definition}
    \begin{block}{Definition: formale Sprache}
        Eine \emph{formale Sprache} (über einem Alphabet $A$) ist eine Teilmenge
$L \subseteq A*$.
    \end{block}
\end{frame}
\begin{frame}{Erklärung}
    \begin{block}{Erklärung}
        $L$ ist also eine Menge. Darin sind alle syntaktisch korrekten Gebilde enthalten.
    \end{block}
\end{frame}
\begin{frame}{Beispiel}
    \begin{block}{Beispiel}
        \begin{enumerate}
            \item Das Alphabet ist $A = \left\{ 0, 1, 2, 3, 4, 5, 6, 7, 8, 9, - \right\}$
            \item Die Sprache $L$ sind alle Dezimalzahlen
                \pause
            \item \invisible<1>{$\Rightarrow -22 \in L$}
                \pause
            \item \invisible<1-2>{$\Rightarrow 22-0---3 \notin L$ (aber $\in A$!)}
        \end{enumerate}
    \end{block}
\end{frame}

\begin{frame}{Produkt}
    \begin{block}{Definition: Produkt}
        Seien $L_1$ und $L_2$ zwei formale Sprachen. Dann bezeichnet
        \begin{align*}
            L_1 \cdot L_2 = \left\{w_1 w_2 | w_1 \in L_1 \text{ und } w_2 \in L_2 \right\}
        \end{align*}
        das Produkt der Sprachen $L_1$ und $L_2$.
    \end{block}
\end{frame}
        
\begin{frame}{Potenzen}
    \begin{block}{Definition: Potenzen}
        $L$ sei eine formale Sprache. Rekursiv lässt sich auch die Potenz davon definieren.
        \begin{align*}
            L^0 &= \left\{ \epsilon\right\}\\
            L^\mathrm{i + 1} &= L^\mathrm{i} \cdot L
        \end{align*}
    \end{block}
\end{frame}
        
\begin{frame}{Konkatenationsabschluss}
    \begin{block}{Definition: Konkatenationsabschluss}
        $L$ sei eine formale Sprache. Dann ist der Konkatenationsabschluss:
        \begin{align*}
            L^* &= \bigcup_\mathrm{i = 0}^\infty L^\mathrm{i}\\
        \end{align*}
        Der $\epsilon$-freie Konkatenationsabschluss ist:
        \begin{align*}
            L^+ &= \bigcup_\mathrm{i = 1}^\infty L^\mathrm{i}
        \end{align*}
    \end{block}
\end{frame}

\begin{frame}{Konkatenationsabschluss}
    \begin{alertblock}{$\epsilon$-freier Konkatenationsabschluss}
        Falls $\epsilon \in L$, so enthält der $\epsilon$-freie Konkatenationsabschluss auch $\epsilon$.
    \end{alertblock}
\end{frame}

\begin{frame}{Beispiele}
    \begin{block}{Beispiele}
        \begin{enumerate}
            \item IP4-Adressen
            \item Programmiersprache C
            \item HTML
            \item E-Mail (RFC 5322)
        \end{enumerate}
    \end{block}
\end{frame}

\begin{frame}{Wie es geht}
    \begin{block}{Beispiel}
        \begin{enumerate}
            \item Alle Wörter, die genau ein "`b"' enthalten
                \pause
            \item Alphabet: $A = \left\{ a, b\right\}$
                \pause
            \item $L = \left\{ a\right\}^* \cdot \left\{ b\right\} \cdot \left\{ a\right\}^*$ oder
            \item $L = \left\{ w_1 b w_2 | w_1, w_2 \in \left{ a\right}^* \right\}$
        \end{enumerate}
    \end{block}
    \pause
    \begin{block}{Übung}
        \begin{enumerate}
            \item Was ist $L^3$?
                \pause
            \item Was ist $L^i \ \left\{ b\right\}^*$?
        \end{enumerate}
    \end{block}
\end{frame}

\section{Aufgaben}
\begin{frame}{Übungsaufgabe}
    \begin{exampleblock}{Winter 2010/2011}
        Es sei $A = \left\{a, b\right\}$. Beschreiben Sie die folgenden formalen Sprachen mit den Symbolen \{, \}, a, b, $\epsilon$, $\bigcup$, ∗, Komma, ), ( und +:
        \begin{enumerate}
            \item die Menge aller Wörter über $A$, die das Teilwort "`ab"' enthalten
            \item die Menge aller Wörter über $A$, deren vorletztes Zeichen ein "`b"' ist
            \item die Menge aller Wörter über $A$, in denen nirgends zwei "`b"'s hintereinander vorkommen
        \end{enumerate}
    \end{exampleblock}
\end{frame}

% \begin{frame}{Übungsaufgabe}
%     \begin{block}{Lösung}
%         \begin{enumerate}
%             \item $\left\{a, b\right\}^∗ \cdot \left\{ab\right\} \cdot \left\{a, b\right\}^∗$
%             \item $\left\{a, b\right\}^∗ \cdot \left\{b\right\} \cdot \left\{a, b\right\}^1$
%             \item $\left\{a, ba\right\}^∗ \cdot \left\{b, \epsilon\right\}$
%         \end{enumerate}
%     \end{block}
% \end{frame}
% \begin{frame}{Übungsaufgabe}
%     \begin{block}{Winter 2008/2009}
%         Es sei A = {a, b}. Die Sprache L ⊂ A∗ sei definiert durch
%         \begin{align*}
%             L = \left( \left\{ a\right\}^* \left\{ b\right\} \left\{ a\right\}^*\right)^*
%         \end{align*}
%         Zeigen Sie, dass jedes Wort w aus $\left\{a, b\right\}^∗$ , das mindestens einmal das Zeichen $b$ enthält, in $L$ liegt. (Hinweis: Führen Sie eine Induktion über die Anzahl der Vorkommen des Zeichens b in w durch.)
%     \end{block}
% \end{frame}
% \begin{frame}{Übungsaufgabe}
%     \begin{block}{Induktionsanfang}
%         Für k = 1: In diesem Fall lässt sich das Wort w aufteilen in
%         \begin{align*}
%             w = w_1 \cdot b \cdot w_2
%         \end{align*}
%         wobei $w_1$ und $w_2$ keine b enthalten und somit in $\left\{ a\right\}^∗$ liegen. Damit gilt $w \in \left\{ a\right\}^∗ \left\{ b\right\}\left\{ a\right\}^∗$ und somit auch
%         \begin{align*}
%             w \in \left(\left\{ a\right\}^* \left\{ b\right\}\left\{ a\right\}^* \right)^* = L
%         \end{align*}        
%     \end{block}
% \end{frame}
% \begin{frame}{Übungsaufgabe}
%     \begin{block}{Induktionsannahme}
%         Für ein festes $k \in \mathbb{N}$ gilt, dass alle Wörter über $\left\{ a, b\right\}^∗$ , die genau k-Mal das Zeichen $b$ enthalten, in $L$ liegen.        
%     \end{block}
%     \begin{block}{Induktionsschritt}
%         Wir betrachten ein Wort $w$ , das genau $k + 1$ Mal das Zeichen "`b"' enthält. Dann kann man $w$ zerlegen in $w = w_1 \cdot w_2$ , wobei $w_1$ genau einmal das Zeichen $b$ enthält und $w_2$ genau k Mal das Zeichen "`b"'. Nach Induktionsanfang liegt $w_1$ in $\left\{ a\right\}^∗ \left\{ b\right\}\left\{ a\right\}^∗$. Nach Induktionsvoraussetzung liegt $w_2$ in $\left(\left\{ a\right\}^∗ \left\{ b\right\}\left\{ a\right\}^∗ \right)^∗$ , was bedeutet, dass $w = w_1 \cdot w_2$ in $\left(\left\{ a\right\}^∗ \left\{ b\right\}\left\{ a\right\}^∗ \left)\left(\left\{ a\right\}^∗ \left\{ b\right\}\left\{ a\right\}^∗ \right)^∗ \subset \left(\left\{ a\right\}^∗ \left\{ b\right\}\left\{ a\right\}^∗ \right)^∗ = L$ liegt und die Behauptung ist gezeigt.
%     \end{block}
% \end{frame}<++>
% 

\end{document}
