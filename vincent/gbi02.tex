%% LaTeX-Beamer template for KIT design
%% by Erik Burger, Christian Hammer
%% title picture by Klaus Krogmann
%%
%% version 2.1
%%
%% mostly compatible to KIT corporate design v2.0
%% http://intranet.kit.edu/gestaltungsrichtlinien.php
%%
%% Problems, bugs and comments to
%% burger@kit.edu

\documentclass[18pt]{beamer}
\usepackage[utf8x]{inputenc}
\usepackage{units}
\usepackage{booktabs}

%% CUSTOM
\usepackage{amsmath}
\usepackage{algpseudocode}

%% Definitions
\DeclareMathOperator{\div2}{div}
\renewcommand{\algorithmicrequire}{\textbf{Input:}}
\renewcommand{\algorithmicensure}{\textbf{Output:}}
\algnewcommand\algorithmicto{\textbf{to}}
\algrenewtext{For}[3]{\algorithmicfor\ $#1 \gets #2$ \algorithmicto\ $#3$ \algorithmicdo}
\algnewcommand\algorithmicod{\textbf{od}}
\algrenewtext{EndWhile}{\algorithmicod}
\algrenewtext{EndFor}{\algorithmicod}
%\AtBeginSection[]{%
%\begin{frame}<beamer> % do nothing in handouts
%    \frametitle{Überblick}
%    \tableofcontents[sectionstyle=show/shaded,
%    subsectionstyle=show/show/hide]
%\end{frame}
%}
%\AtBeginSubsection[]{%
%\begin{frame}<beamer> % do nothing in handouts
%    \frametitle{Überblick}
%    \tableofcontents[sectionstyle=show/shaded,
%    subsectionstyle=show/shaded/hide]
%\end{frame}
%}

%% SLIDE FORMAT

% use 'beamerthemekit' for standard 4:3 ratio
% for widescreen slides (16:9), use 'beamerthemekitwide'

\usepackage{templates/beamerthemekit}
%\usepackage{templates/beamerthemekitwide}

 %% TITLE PICTURE

 % if a custom picture is to be used on the title page, copy it into the 'logos'
 % directory, in the line below, replace 'mypicture' with the 
 % filename (without extension) and uncomment the following line
 % (picture proportions: 63 : 20 for standard, 169 : 40 for wide
 % *.eps format if you use latex+dvips+ps2pdf, 
 % *.jpg/*.png/*.pdf if you use pdflatex)


 \titleimage{banner}
 
 
%% Define some colors:
\definecolor{darkblue}{rgb}{0,0,.5}
\definecolor{darkgreen}{rgb}{0,.5,0}

 %% TITLE LOGO

 % for a custom logo on the front page, copy your file into the 'logos'
 % directory, insert the filename in the line below and uncomment it

\titlelogo{logo_150x150}
 
 % (*.eps format if you use latex+dvips+ps2pdf,
 % *.jpg/*.png/*.pdf if you use pdflatex)
 
 %% TikZ INTEGRATION
 
 % use these packages for PCM symbols and UML classes
 % \usepackage{templates/tikzkit}
 % \usepackage{templates/tikzuml}
 
 % the presentation starts here
 
\author{Dominik Muth - dominik.muth@student.kit.edu}
\institute{Institut f\"ur Informatik}

\subtitle{Foliensatz 02}
\date{1. November 2012}
\begin{document}

\begin{frame}
    \titlepage
\end{frame}

\begin{frame}{Outline/Gliederung}
    \tableofcontents
\end{frame}

\section{Besprechung des 1. Übungsblattes}

\section{Wörter und Alphabete}
\begin{frame}{Alphabete}
    \begin{block}{Definition}
        Ein Alphabet ist eine endliche, nichtleere Menge an "`Zeichen"' oder "`Symbolen"'.
    \end{block}
    \begin{exampleblock}{Beispiele}
        \begin{itemize}
            \item $A = \left\{a, b, d\right\}$
            \item $B = \left\{3, 9, V, k\right\} $
            \item Der ASCII-Zeichensatz
        \end{itemize}
    \end{exampleblock}
\end{frame}

\begin{frame}{Wörter}
    \begin{block}{Definition}
        Ein Wort über einem Alphabet A ist eine Folge von Zeichen aus A.
    \end{block}
    \begin{exampleblock}{Beispiele}
        \begin{itemize}
            \item $A = \left\{H, a, l, o, \textvisiblespace, W, e, t\right\}$  enthält das Wort\\
            Hallo Welt
            \item \dots
        \end{itemize}
    \end{exampleblock}
\end{frame}

\begin{frame}{Menge aller Wörter}
    \begin{block}{Definition}
        Die Menge aller Wörter über einem Alphabet $A$ sind alle Wörter, in denen nur Zeichen aus $A$ enthalten sind. Dies wird als $A^*$ geschrieben.
        \end{block}
    \begin{exampleblock}{Beispiele}
        Alphabet set $A = \left\{a, b \right\}$, dann enthält $A*$:
        \begin{itemize}
            \item a
            \item b
            \item aa
            \item ab
            \item ba
            \item \dots
        \end{itemize}
    \end{exampleblock}
\end{frame}

\begin{frame}{Konkatenation von Wörtern}
    \begin{block}{Definition}
        Die Konkatenation zweier Worte $w_1$ und $w_2$ aus den Alphabeten $A$ und $B$ wird geschrieben als $w_1 \circ w_2 \in \left( A \cup B\right)$    
    \end{block}
    \begin{exampleblock}{Beispiele}
        \begin{itemize}
            \item $A = \left\{B, e, t\right\}$  enthält das Wort $w_1 = Bett$\\
            \item $B = \left\{w, a, n, z, e\right\}$ enthält das Wort $w_2 = wanze$\\
                \pause
            \item $w_1 \circ w_2 = Bettwanze \neq w_2 \circ w_1 = wanzeBett$
                \pause
            \item $A \cup B = \left\{B, e, t, w, a, n, z\right\}$ ($e$ nur einmal!)
        \end{itemize}
    \end{exampleblock}
\end{frame}

\begin{frame}{Mehrfachkonkatenation}
    \begin{exampleblock}{Beispiel}
        $w$ sei ein Wort (zum Beispiel über dem vorherigen Alphabet $A$).
        \begin{itemize}
            \item $w = Bett$
            \item $w^3 = BettBettBett$
        \end{itemize}
    \end{exampleblock}
\end{frame}

\begin{frame}{Das leere Wort}
    \begin{block}{Definition}
        Das leere Wort wird mit $\epsilon$ geschrieben und hat die Länge $0$.
    \end{block}
    \begin{exampleblock}{Beispiele}
        Das leere Wort ist \emph{nicht das Leerzeichen}. 
        \begin{itemize}
            \item $\epsilon \circ w \circ \epsilon = w$
            \item $w^0 = \epsilon$
        \end{itemize}
    \end{exampleblock}
\end{frame}

\begin{frame}{Wortlänge}
    \begin{block}{Definition}
        Die Länge eines Wortes $w$ gibt die Anzahl der darin enthaltenen Zeichen an. Gekennzeichnet wird dies mit dem "`Pipe-Symbol"' $\left| w \right|$.
    \end{block}
    \begin{exampleblock}{Beispiel}
        \begin{itemize}
            \item $\left| Hallo\right| = 5$
                \pause
            \item $\left| w^k\right| = k\cdot \left| w\right|$
                \pause
            \item $\left| \epsilon\right| = 0$
                \pause
            \item $\left| w_1 \circ w_2 \right| = \left| w_1\right| + \left| w_2\right|$
        \end{itemize}
    \end{exampleblock}
\end{frame}

\begin{frame}{Präfix und Suffix}
    \begin{block}{Definition: Präfix}
        Ein Präfix ist ein beliebig langer Teil am Anfang eines Wortes. $a$ ist ein Präfix von $w$, falls gilt: $w = a \circ b$.
    \end{block}
    \pause
    \begin{block}{Definition: Suffix}
        Ein Suffix ist ein beliebig langer Teil am Ende eines Wortes. $b$ ist ein Suffix von $w$, falls gilt: $w = a\circ b$.
    \end{block}
\end{frame}

\begin{frame}{Aufgaben}
    \begin{exampleblock}{Aufgabe}
        Gegeben sei das Alphabet $A = \{ 0, 1\}$.
        \begin{itemize}
            \item Welche Worte befinden sich in $A^5$?
                \pause
            \item Ist auch das leere Wort darin enthalten?
                \pause
            \item Was ist der Unterschied zwischen $A^2 \times A^2$ und $A^2 \cdot A^2$?
        \end{itemize}
    \end{exampleblock}
\end{frame}

\section{Vollständige Induktion}
\begin{frame}{Vollständige Induktion}
    \begin{block}{Definition}
        Die vollständige Induktion ist eine mathematische Beweismethode, mit der die Gültigkeit einer Aussage für alle natürlichen Zahlen bewiesen werden kann.
    \end{block}
\end{frame}
\begin{frame}{Vollständige Induktion}
    \begin{block}{Vorgehen}
        Eine Behauptung ist gegeben.\\\pause
        Die vollständige Induktion besteht aus drei Schritten:
        \begin{enumerate}
            \item \emph{Induktionsanfang IA}: Zeige die Gültigkeit der Behauptung für das erste Element.
                \pause
            \item \emph{Induktionsvoraussetzung IV}: Wir wissen, dass die Behauptung für ein beliebiges, aber festes Element $n$ gilt.
                \pause
            \item \emph{Induktionsschritt IS}: Prüfe die Gültigkeit für ein darauffolgendes Element $n+1$.
        \end{enumerate}
    \end{block}
\end{frame}

\begin{frame}{Beispiel}
    \begin{block}{Behauptung}
        $ 1 + 2 + 3 + \dots + n = \frac{ n\cdot \left( n + 1\right) }{2}$\\
    \end{block}
    \pause
    \begin{block}{Beweis}
        \begin{enumerate}
            \item \emph{IA}: $n = 1$: Oben einsetzen, passt: $1 = 1$.
                \pause
            \item \emph{IV}: "`Es gibt ein beliebiges, aber festes $n$ für das die obige Behauptung gilt."' Dieses $n$ möchte ich nun $k$ nennen, einfach so :-)
                \pause
            \item \emph{IS}: $k \rightarrow k + 1$: 
                \begin{itemize}
                    \item Links: $ 1 + 2 + \dots + k + \left( k+1\right) \overset{\text{IV}}{=} \frac{k\cdot \left( k + 1\right)}{2} + \left( k + 1\right)$
                    \item Rechts: $\frac{\left(k + 1\right) \cdot \left(\left( k + 1\right) + 1\right)}{2} = \frac{\left( k + 1\right) \cdot \left( k + 2\right)}{2} = \frac{\left( k + 1\right) \cdot k}{2}  + \frac{\left( k + 1\right)\cdot 2}{2}$
                \end{itemize}
                \pause
            \item Die Behauptung stimmt.
        \end{enumerate}
    \end{block}
\end{frame}

\begin{frame}{Übung}
    \begin{block}{Behauptung}
        $\forall n \in \mathbb{N}_0 : x_{\mathrm{n + 1}} = x_\mathrm{n} + 2 \wedge x_0 = 0 \Leftrightarrow x_\mathrm{n} = 2\, n$
    \end{block}
    \pause
    \begin{block}{Lösung}
        \begin{enumerate}
            \item \emph{IA}: $n = 0$. \invisible<1>{$x_0 = 0$ (nach Vorgabe) und $2\cdot 0 = 0$ (die rechte Seite)}
                \pause
            \item \emph{IV}: \invisible<1-2>{"`Für ein beliebiges, aber festes $n$ gilt die obige Behauptung:$x_\mathrm{n} = 2\, n$"'}
                \pause
            \item \emph{IS}:\\ \invisible<1-3>{Rechte Seite: $x_\mathrm{n + 1} = 2\left( n + 1\right)$.\\
                Linke Seite: $x_\mathrm{n + 1} = x_\mathrm{n} + 2  \overset{\text{IV}}{=} 2\, n + 2$}
        \end{enumerate}
    \end{block}
\end{frame}

\end{document}
