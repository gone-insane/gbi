\include{header}
\subtitle{Foliensatz 10}
\date{10. Januar 2013}

\begin{document}

\begin{frame}
    \titlepage
\end{frame}

\begin{frame}{Outline/Gliederung}
    \tableofcontents
\end{frame}

\section{Master-Theorem}
\begin{frame}{Master-Theorem}
    \begin{block}{Definition}
        Für einen \emph{rekursiven} Algorithmus der Form
        \begin{align*}
            T\left( n \right) = aT\left( \frac{n}{b} \right) + f\left( n \right)
        \end{align*}
        kann die Laufzeit für drei Fälle abgeschätzt werden:
        \pause
        \begin{enumerate}
            \item Wenn $f\left( n \right)\n \mathcal{O}\left( n^{\log_b a-\varepsilon} \right)$ für ein $\varepsilon > 0$, dann ist $T\left( n \right) \in \Theta\left( n^\log_b a \right)$
                \pause
            \item Wenn $f\left( n \right)\n \Theta\left( n^{\log_b a} \right)$, dann ist $T\left( n \right) \in \Theta\left( n^\log_b a \log n\right)$
                \pause
            \item Wenn $f\left( n \right)\n \Omega\left( n^{\log_b a+\varepsilon} \right)$ für ein $\varepsilon > 0$, \\ und wenn es eine Konstante $d$ gibt mit $0<d<1$,\\ sodass für alle hinreichend großen $n$ gilt $af\left( \nicefrac{n}{b} \right) \leq df\left( n \right)$,\\ dann ist $T\left( n \right) \in \Theta\left( f\left( n \right) \right)$
        \end{enumerate}
    \end{block}
    \begin{itemize}
        \item Fall 2 wird etwa bei Quicksort benötigt
        \item Fall 3 ist eher die Ausnahme
    \end{itemize}
\end{frame}

\section{Mealy-Automat}
\begin{frame}{Mealy-Automat}
    \begin{block}{Definition: Mealy-Automat}
        Der Mealy-Automat $A = \left( Z, z_0, X, f, Y, g \right)$ besteht aus
        \begin{enumerate}
            \item der endlichen Zustandsmenge $\mathbf{Z}$,
            \item dem Startzustand $\mathbf{z_0}$,
            \item dem Eingabealphabet $\mathbf{X}$,
            \item der Zustandsübergangsfunktion $\mathbf{f: Z\times X \rightarrow Z}$,
            \item einem Ausgabealphabet $\mathbf{Y}$ und
            \item der Ausgabefunktion $\mathbf{g: Z\times X \rightarrow Y^*}$.
        \end{enumerate}
    \end{block}
\end{frame}
\begin{frame}{Getränkeautomat}
    \begin{figure}[htbp]
        \centering
        \includegraphics[width=\textwidth,height=\textheight,keepaspectratio]{graphics/10/getraenke.png}
    \end{figure}
\end{frame}
\begin{frame}{Getränkeautomat}
    Was ist was?
    \begin{itemize}
        \item Zustandsmenge $Z$: $\left\{ \left( 0,- \right), \left( 0,R \right), \left( 0,Z \right), \left( 1,- \right), \left( 1,R \right), \left( 1,Z \right) \right\}$
    \end{itemize}
\end{frame}

\section{Moore-Automat}

\section{Endliche Akzeptoren}
\end{document}
