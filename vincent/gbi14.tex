\include{header}
\subtitle{Foliensatz 14}
\date{7. Februar 2013}

\newcommand{\sq}{$\square$}
\newcommand{\da}{$\downarrow$}
\DeclareMathOperator{\cod}{cod}

\begin{document}

\begin{frame}
    \titlepage
\end{frame}

\begin{frame}{Outline/Gliederung}
    \tableofcontents
\end{frame}

\section{Statistik}
\begin{frame}{Statistiken}
    Die folgenden Grafiken beziehen sich
    \begin{itemize}
        \item bei den Übungsblättern auf diejenigen, die den Übungsschein erhalten haben und
        \item bei der Übungsklausur auf diejenigen, die abgegeben haben.
    \end{itemize}
\end{frame}
\begin{frame}{Verlauf des Punktestands}
    \includegraphics[scale=0.4]{graphics/14/punkte1.pdf}
\end{frame}
\begin{frame}{Verlauf des Punktestands}
    \includegraphics[scale=0.35]{graphics/14/punkte2.pdf}\\
    blau: Mittelwert, rot: Median
\end{frame}
\begin{frame}{Punkteverteilung in der Probleklausur}
    \includegraphics[scale=0.9]{graphics/14/probeklausur1.pdf}
\end{frame}
\section{Wiederholung}
\begin{frame}{Wiederholung - Quiz}
    \begin{itemize}
        \item Jedes Problem kann von einer Turingmaschine entschieden werden. \visible<2>{Nein.}
        \item Warum? \visible<3>{Siehe Halteproblem.}
        \item Gilt $\mathrm{P} \subset \mathrm{PSPACE}$? \visible<4>{Ja.}
        \item Gibt es endliche Akzeptoren für Sprachen $L$, die weniger Zustände haben als $L$ Nerode-Äquivalenzklassen? \visible<5>{Nein.}
    \end{itemize}
\end{frame}
\section{Kongruenzrelationen}
\begin{frame}{Verträglichkeit}
    \begin{block}{Definition: Verträglichkeit}
        Es sei $\equiv$ eine Äquivalenzrelation auf einer Menge $M$ und $f: M\rightarrow M$ eine Abbildung. Man sagt, dass $\equiv$ mit $f$ verträglich ist, wenn für alle $x_1, x_2 \in M$ gilt:
        \begin{align*}
            x_1 \equiv x_2 \Longrightarrow f\left( x_1 \right) \equiv f\left( x_2 \right)
        \end{align*}
    \end{block}
    Was bedeuted das anschaulich? Fallen euch Beispiele ein?
\end{frame}
\begin{frame}{Beispiel modulo n}
    Wir kennen noch vom letzten Mal:
    \begin{align}
        x_1 \equiv x_2 \left( \mod n \right) \Leftrightarrow x_1 -x_2 = kn\label{eq1}
    \end{align}
    \pause
    Das heißt auch, dass $x_1$ und $x_2$ bei einer Division mit $n$ den gleichen Rest haben.
    \pause
    \begin{align}
        y_1 \equiv y_2 \left( \mod n \right) \Leftrightarrow y_1 - y_2 = mn\label{eq2}
    \end{align}
    \pause
    Ich behaupte es gilt:
    \begin{align}
        x_1 + y_1 \equiv x_2  + y_2 \left( \mod n \right)\label{rel1}\\
        x_1 \cdot y_1 \equiv x_2 \cdot y_2 \left( \mod n \right)\label{rel2}
    \end{align}
    Beweis.
\end{frame}
\begin{frame}{Beweis von \eqref{rel1}}
    Addieren nun die beiden Gleichungen \eqref{eq1} und \eqref{eq2}:
    \pause
    \begin{align*}
        \left( x_1 + y_1 \right) - \left( x_2 + y_2 \right) = \left( x_1 - x_2 \right) + \left( y_1 - y_2 \right) = \left( k + m \right) n
    \end{align*}
    \pause
    Und wir sehen, dass gilt
    \begin{align*}
        x_1 + y_1 \equiv x_2 + y_2 \left( \mod n \right)
    \end{align*}
\end{frame}
\begin{frame}{Beweis von \eqref{rel2}}
    Löse Gleichung~\eqref{eq1} nach $x_1$ auf und Gleichung~\eqref{eq2} nach $y_1$ und multipliziere beide Seiten:
    \begin{align*}
        x_1 \cdot y_1 &= \left( x_2 + kn \right) \cdot \left( y_2 + mn \right)\\
                        &= x_2 \cdot y_2 + n\left( mx_2 + ky_2 + kmn \right)\\
        x_1 \cdot y_1 - x_2 \cdot y_2  &= n\left( mx_2 + ky_2 + kmn \right)\\
            \Longleftrightarrow x_1\cdot y_1 &\equiv x_2 \cdot y_2 \left( \mod n \right)
    \end{align*}
\end{frame}
\begin{frame}{Kongruenz}
    Damit können wir auch ``nur mit Repräsentanten'' der Äquivalenzklasse rechnen:
    \begin{align*}
        [2] + [3] = [2 + 3] = [5] = [0]\\
        [2] \cdot [3] = [2 \cdot 3] = [6] = [1]\\
    \end{align*}
    Nennt weitere Beispiele für die Äquivalenzrelation Kongruent Modulo $i$, wobei sich $i$ bei jedem von euch erhöht.
\end{frame}
\begin{frame}{Verträglichkeit und Kongruenzrelationen}
    Die Operationen $+$ und $\cdot$ sind also verträglich mit unserer Relation ``kongruent modulo n''.\\
    \begin{block}{Definition: Kongruenzrelation}
        Eine Funktion, die mit allen gerade interessierenden Funktionen oder/und Operationen verträgich ist, nennt man auch \emph{Kongruenzrelation}.
    \end{block}
\end{frame}
\begin{frame}{Kongruenz und die Nerode-Äquivalenz}
    Gegeben sei die Funktion:
    \begin{align*}
        f_x': A_{/\equiv_L}^* \rightarrow A_{/\equiv_L}^*: [w]\mapsto [wx]
    \end{align*}
    Warum ist die Nerode-Äquivalenz mit dieser Abbildung verträglich?
    \visible<2>{Gegeben sei $w_1\equiv_L w_2$, das heißt nach Definition der Nerode-Äquivalenz:
        \begin{align*}
            w_1 w \in L &\Leftrightarrow w_2 w \in L\\
            \left( w_1 x \right) v \in L &\Leftrightarrow w_1\left( x v \right) \in L\\
            &\Leftrightarrow w_2\left( x v \right) \in L \text{ weil } w_1 \equiv_L w_2\\
            &\Leftrightarrow \left( w_2 x \right)v \in L
        \end{align*}
    Damit ist gezeigt, dass die Nerode-Äquivalenz mit der Konkatenation verträglich ist.}
\end{frame}
\begin{frame}{Induzierte Operationen}
    Wir wissen nun, dass $f_x : A^*:w\mapsto wx$ (Konkatenation) mit $\equiv_L$ (Nerode-Äquivalenz) verträglich ist. Damit gilt auch:
    \pause
    \begin{align*}
        f'_x : A_{/\equiv_L}^* \rightarrow A_{/\equiv_L}^*:[w]\mapsto[wx]
    \end{align*}
    Was heißt das?\\
    \pause
    Die Nerode-Äquivalenz ist mit der Konkatenation verträglich. Damit ergibt sich auch eine Relation auf die Äquivalenzmenge $A_{/\equiv_L}^*$. (Die obige Funktion ist \emph{wohldefiniert}.)\\
    \pause
    Damit können wir uns einen endlichen Akzeptor konstruieren, wähle:
    \begin{itemize}
        \item $z_0 = [\varepsilon]$ (Startzustand) und
        \item $F = \left\{ [w] | w\in L \right\}$ (akzeptierte Zustände)
    \end{itemize}
\end{frame}
\section{Halbordnungen}
\begin{frame}{Antisymmetrie}
    \begin{block}{Definition: Antisymmetrie}
        Eine Relation $R\subseteq M\times M$ heißt \emph{antisymmetrisch}, wenn für alle $x, y \in M$ gilt:
        \begin{align*}
            xRy \wedge yRx \Rightarrow x=y
        \end{align*}
    \end{block}
    \pause
    \begin{block}{Definition: Halbordnung}
        Eine Relation $R \subseteq M\times M$ heißt \emph{Halbordnung}, wenn sie
        \begin{itemize}
            \item reflexiv,
            \item antisymmetrisch und
            \item und transitiv
        \end{itemize}
        ist.
    \end{block}
\end{frame}
\begin{frame}{Beispiel: Halbordnung}
    Gegeben seien $v, w \in A^*$.
    \begin{align*}
        w_1 \sqsubseteq_p w_2 \Longleftrightarrow \exists u \in A^*:w_1u = w_2
    \end{align*}
    Was bedeuted das?
    \pause
    Wie kann gezeigt werden, dass das eine Halbordnung ist?
    \visible<2->{Nachzuweisen sind:
        \begin{itemize}
            \item Reflexivität
            \item Antisymmetrie
            \item Transitivität
        \end{itemize}}
\end{frame}
\begin{frame}{Halbordnung $\sqsubseteq_p$ Beweis 1: Reflexivität}
    \begin{align*}
        v\sqsubseteq_p v \iff \exists u \in A^*:vu = v \Rightarrow u = \varepsilon
    \end{align*}
\end{frame}
\begin{frame}{Halbordnung $\sqsubseteq_p$ Beweis 2: Antisymmetrie}
    \begin{align*}
        & v \sqsubseteq_p w \wedge w \sqsubseteq_p v\\
        \iff & \exists u_1 \in A^*:vu_1 = w \wedge \exists u_2 \in A^*: wu_2 = v\\
    \end{align*}
    \begin{align*}
        &\Rightarrow  wu_1u_2 = w  \\
        &\Rightarrow  u_1u_2 = \varepsilon = u_1 = u_2 \\
        &\Rightarrow  v = w 
    \end{align*}
\end{frame}
\begin{frame}{Halbordnung $\sqsubseteq_p$ Beweis 3: Transitivität}
    \begin{align*}
        & v \sqsubseteq_p w \wedge w \sqsubseteq_p x\\
        \iff & \exists u_1 \in A^*:vu_1 = w \wedge \exists u_2 \in A^*: wu_2 = x\\
    \end{align*}
    \begin{align*}
        \Rightarrow & vu_1u_2 = x  \\
        \overset{y=u_1u_2}{\iff} & \exists y \in A^*: vy = x \\
        \iff & v \subseteq_p x
    \end{align*}
\end{frame}
\begin{frame}{Beispiel}
    Ist dies eine korrekte Halbordnung auf $A^*$?
    \begin{align*}
        w_1 \sqsubseteq w_2 \iff \left| w_1 \right| \leq \left| w_2 \right|
    \end{align*}
    \pause
    Test:
    \begin{itemize}
        \item Reflexiv? \visible<2->{Ja.}
        \item Transitiv? \visible<3->{Ja.}
        \item Antisymmetrisch? \visible<4->{Nein.}
    \end{itemize}
\end{frame}
\begin{frame}{Übung: Beweise}
    Zeigen Sie, dass $\leq$ und $\subseteq$ Halbordnungen sind.\\
    Tipp: Diese Definitionen könnten helfen:
    \begin{align*}
        a\leq b&\iff \exists \alpha \in \mathbb{R}_0^+ : a + \alpha = b\\
        A\subseteq B&\iff \left( A \subset B \right) \vee \left( A = B \right)
    \end{align*}
    An der Tafel.
\end{frame}
\begin{frame}{Ordnungen}
    \begin{block}{Definition: Ordnung}
        Eine Relation $R\subseteq M\times M$ ist eine (totale) Ordnung, wenn $R$ eine Halbordnung ist und außerdem gilt:
        \begin{align*}
            \forall x,y \in M: xRy \vee yRx
        \end{align*}
    \end{block}
    Anschaulich: Wenn man sich zwei beliebige Elemente aus $M$ auswählt, stehen diese in Relation zueinander.
    Mehr zum Thema im Skript.
\end{frame}
\begin{frame}{Potenzmenge}
    \begin{block}{Definition: Potenzmenge}
        Die \emph{Potenzmenge} $\mathcal{P}\left( X \right)$ ist die Menge aller Teilmengen von der Grundmenge $X$.
        \begin{align*}
            \mathcal{P}\left( X \right) = \left\{ U \big| U\subseteq X \right\}
        \end{align*}
        Gelegentlich wird auch geschrieben $\mathcal{P}\left( X \right) = 2^X$.
    \end{block}
    Die Potenzmenge besitzt die Mächtigkeit (Anzahl der Elemente):
    \begin{align*}
        \left| \mathcal{P}\left( X \right)\right| = 2^{\left| X\right|}
    \end{align*}
    Beispiel:
    \begin{align*}
        \mathcal{P}\left( \left\{ a,b \right\} \right) = \left\{ \emptyset, \left\{ a \right\}, \left\{ b \right\}, \left\{ a,b \right\} \right\}
    \end{align*}
\end{frame}
\begin{frame}{Halbordnung $\subseteq$ auf einer Potenzmenge $\mathcal{P}\left( \left\{ a,b,c \right\} \right)$}
    \begin{figure}
    \centering
    \begin{tikzpicture}
        \tikzstyle{every node}=[font=\tiny]
        \node[state] (max) at (2,4) {$\left\{ a,b,c\right\}$};
        \node[state] (a) at (-2,2) {$\left\{ a,b\right\}$};
        \node[state] (b) at (2,2) {$\left\{ a,c\right\}$};
        \node[state] (c) at (4,2) {$\left\{ b,c\right\}$};
        \node[state] (d) at (-2,0) {$\left\{ a\right\}$};
        \node[state] (e) at (0,0) {$\left\{ b\right\}$};
        \node[state] (f) at (4,0) {$\left\{c\right\}$};
        \node[state] (min) at (0,-2) {$\left\{  \right\}$};
        \path[->] (min) edge (d)
             (d) edge (a) 
             (a) edge (max)
             (b) edge (max)
             (f) edge (b)
             (min) edge (e)
             (min) edge (f)
             (f) edge (c)
             (c) edge (max)
             (d) edge (b)
             (e) edge (a)
             (e) edge (c)
             (min) edge[loop right] (max)
             (max) edge[loop left] (max)
             (a) edge[loop left] (a)
             (b) edge[loop right] (b)
             (c) edge[loop right] (c)
             (d) edge[loop left] (d)
             (e) edge[loop left] (e)
             (f) edge[loop right] (f)
             (d) edge (max)
             (e) edge (max)
             (f) edge (max)
             (min) edge (max)
             (min) edge (a)
             (min) edge (b)
             (min) edge (c)
             (min) edge (d);
    \end{tikzpicture}
    \end{figure}
\end{frame}
\section{Hasse-Diagramm}
\begin{frame}{Hasse-Diagramm}
    \begin{block}{Definitoin: Hasse-Diagramm}
        Ein Diagramm einer Halbordnung $\sqsubseteq$ auf einer Menge $M$ heißt \emph{Hasse-Diagramm}, wenn es im Diagramm eine Kante von $a$ nach $b$ ($a,b \in M$), wenn gilt
        \begin{align*}
            \nexists c \in M: a\sqsubseteq c \sqsubseteq b\\
            A \neq b
        \end{align*}
    \end{block}
    Anschaulich gesprochen, werden nur die ``notwendigen'' Kanten eingezeichnet. Schlaufen und Kanten, die sich durch Transitivität ergeben, werden weggelassen.
\end{frame}
\begin{frame}{Beispiel: Hasse-Diagramm}
    \begin{figure}
    \centering
    \begin{tikzpicture}
        \tikzstyle{every node}=[font=\tiny]
        \node[state] (max) at (2,4) {$\left\{ a,b,c\right\}$};
        \node[state] (a) at (-2,2) {$\left\{ a,b\right\}$};
        \node[state] (b) at (2,2) {$\left\{ a,c\right\}$};
        \node[state] (c) at (4,2) {$\left\{ b,c\right\}$};
        \node[state] (d) at (-2,0) {$\left\{ a\right\}$};
        \node[state] (e) at (0,0) {$\left\{ b\right\}$};
        \node[state] (f) at (4,0) {$\left\{c\right\}$};
        \node[state] (min) at (0,-2) {$\left\{  \right\}$};
        \path[->] (min) edge (d)
             (d) edge (a) 
             (a) edge (max)
             (b) edge (max)
             (f) edge (b)
             (min) edge (e)
             (min) edge (f)
             (f) edge (c)
             (c) edge (max)
             (d) edge (b)
             (e) edge (a)
             (e) edge (c);
    \end{tikzpicture}
    \end{figure}
\end{frame}
\begin{frame}{Extreme Elemente im Hasse-Diagram}
    \begin{block}{Definition: extremale Elemente im Hasse-Diagram}
        Es sei $\left( M, \sqsubseteq \right)$ eine halbgeordnete Menge und $T\subseteq M$. Ein Element $x\in T$ heißt
        \begin{itemize}
            \item minimales Element von $T$, wenn es kein $y\in T, y\neq x$ gibt, mit $y\sqsubseteq x$
            \item maximales Element von $T$, wenn es kein $y\in T, y\neq x$ gibt, mit $x\sqsubseteq y$
            \item größtes Element von $T$, wenn für alle $y\in T$ gilt $y\sqsubseteq x$
            \item kleinstes Element von $T$, wenn für alle $y\in T$ gilt $x\sqsubseteq y$
        \end{itemize}
    \end{block}
    Eine Teilmenge $T$ kann mehrere minimale (bzw. maximale) Elemente besitzen, aber nur ein kleinstes (bzw. größtes) Element.
\end{frame}
\begin{frame}{Klausuraufgabe: Winter 2010/2011}
    \begin{exampleblock}{}
        Geben Sie das Hasse-Diagramm einer Halbordnung auf einer dreielementigen Menge an, die genau zwei maximale und zwei minimale Elemente besitzt.
    \end{exampleblock}
    \visible<2->{Dreielementige Menge: $\left\{ 1, 2, 3 \right\}$\\}
    \visible<3->{
        \begin{tikzpicture}
            \node[state] (1) at (0,0) {1};
            \node[state] (2) at (0,2) {2};
            \node[state] (3) at (2,2) {3};
            \path[->] (1) edge (2);
        \end{tikzpicture}
    }
\end{frame}
\end{document}
