%% LaTeX-Beamer template for KIT design
%% by Erik Burger, Christian Hammer
%% title picture by Klaus Krogmann
%%
%% version 2.1
%%
%% mostly compatible to KIT corporate design v2.0
%% http://intranet.kit.edu/gestaltungsrichtlinien.php
%%
%% Problems, bugs and comments to
%% burger@kit.edu

\documentclass[18pt]{beamer}
\usepackage[utf8x]{inputenc}
\usepackage{units}
\usepackage{booktabs}
\usepackage{amsmath}
\usepackage{algpseudocode}

%% Definitions
\DeclareMathOperator{\div2}{div}
\renewcommand{\algorithmicrequire}{\textbf{Input:}}
\renewcommand{\algorithmicensure}{\textbf{Output:}}
\algnewcommand\algorithmicto{\textbf{to}}
\algrenewtext{For}[3]{\algorithmicfor\ $#1 \gets #2$ \algorithmicto\ $#3$ \algorithmicdo}
\algnewcommand\algorithmicod{\textbf{od}}
\algrenewtext{EndWhile}{\algorithmicod}
\algrenewtext{EndFor}{\algorithmicod}
\AtBeginSection[]{%
\begin{frame}<beamer> % do nothing in handouts
    \frametitle{Überblick}
    \tableofcontents[sectionstyle=show/shaded,
    subsectionstyle=show/show/hide]
\end{frame}
}
\AtBeginSubsection[]{%
\begin{frame}<beamer> % do nothing in handouts
    \frametitle{Überblick}
    \tableofcontents[sectionstyle=show/shaded,
    subsectionstyle=show/shaded/hide]
\end{frame}
}


%% SLIDE FORMAT

% use 'beamerthemekit' for standard 4:3 ratio
% for widescreen slides (16:9), use 'beamerthemekitwide'

\usepackage{templates/beamerthemekit}
%\usepackage{templates/beamerthemekitwide}

%% TITLE PICTURE

% if a custom picture is to be used on the title page, copy it into the 'logos'
% directory, in the line below, replace 'mypicture' with the 
% filename (without extension) and uncomment the following line
% (picture proportions: 63 : 20 for standard, 169 : 40 for wide
% *.eps format if you use latex+dvips+ps2pdf, 
% *.jpg/*.png/*.pdf if you use pdflatex)

%\titleimage{mypicture}

%% TITLE LOGO

% for a custom logo on the front page, copy your file into the 'logos'
% directory, insert the filename in the line below and uncomment it

%\titlelogo{mylogo}
\titlelogo{empty_logo}

% (*.eps format if you use latex+dvips+ps2pdf,
% *.jpg/*.png/*.pdf if you use pdflatex)

%% TikZ INTEGRATION

% use these packages for PCM symbols and UML classes
% \usepackage{templates/tikzkit}
% \usepackage{templates/tikzuml}

% the presentation starts here

\title[GBI Tutorium]{GBI Tutorium Nr. }
\subtitle{Foliensatz 0333}
\date{6. November 2012}
\author{Vincent Hahn -- vincent.hahn@student.kit.edu}

\institute{}

% Bibliography

%\usepackage[citestyle=authoryear,bibstyle=numeric,hyperref,backend=biber]{biblatex}
%\addbibresource{templates/example.bib}
%\bibhang1em

\begin{document}

% change the following line to "ngerman" for German style date and logos, english: english
\selectlanguage{ngerman}

\begin{frame}
    \titlepage
\end{frame}

\begin{frame}{Outline/Gliederung}
    \tableofcontents
\end{frame}

\section{Division mit Rest}

\begin{frame}{Division mit Rest}
    \begin{block}{Definition}
        \begin{align*}
            &\forall x \in \mathbb{N}_0, \forall y \in \mathbb{N}_+:\\
            &x = y\cdot\left( x \div2 y\right) + \left( x\mod y\right)
        \end{align*}
        Hierbei ist $\div2$ die Ganzzahldivision ohne Rest.
    \end{block}
    \pause
    \begin{exampleblock}{Beispiel}
        Den Rest $a$ der Ganzzahldivision erhält man also mit $a = x \mod y$:
        \begin{align*}
            1 = 4 \mod 3
        \end{align*}
    \end{exampleblock}
\end{frame}

\begin{frame}{Division mit Rest}
    \begin{block}{Folgerung}
        Aus der Definition kann direkt geschlossen werden:
        \begin{align*}
            x \div2 y &\in \mathbb{N}_0\\
            x \mod y &\in \left\{ 0, \dots , y-1\right\}
        \end{align*}
    \end{block}
\end{frame}

\begin{frame}{Übung}
    % Diese Übung eignet sich sehr gut für ein Blitzlicht
    \begin{exampleblock}{mündlich}
        \begin{table}
            \begin{tabular}{llrr}
                \toprule
                $x$ & $y$ & $x \div2 y$ & $x \mod y$\\
                \midrule
                4 & 3 & \invisible<1>{1 & 1}\\
                2 & 1 & \invisible<1-2>{2 & 0}\\
                10 & 3 & \invisible<1-3>{3 & 1}\\
                8 & 3 & \invisible<1-4>{2 & 2}\\
                9 & 2 & \invisible<1-5>{4 & 1}\\
                4 & 3 & \invisible<1-6>{1 & 1}\\
                \bottomrule
            \end{tabular}
        \end{table}
        \pause\pause\pause\pause\pause\pause
    \end{exampleblock}
\end{frame}

\begin{frame}{Größter gemeinsamer Teiler}
    \begin{block}{Definition}
        Der größte gemeinsame Teiler zweier Zahlen ist die größtmögliche Zahl $m \in \mathbb{N}_0$, für die gilt:
        \begin{align*}
            a \div2 m = 0 \wedge b \ðiv m = 0
        \end{align*}
    \end{block}
    \pause
    \begin{exampleblock}{Bestimmung}
        Der größte gemeinsame Teiler kann mit Primfaktorzerlegung bestimmt werden:
        \begin{align*}
            a = 3528, b = 3780\\
            \Rightarrow a = 2^3\cdot 3^2\cdot 5^0 \cdot 7^2\\
            \Rightarrow b = 2^2\cdot 3^3\cdot 5^1\cdot 7^1\\
        \end{align*}
        Damit ist der \emph{ggT} $2^2 \cdot 3^2\cdot 5^0\cdot 7^1 = 252$
    \end{exampleblock}
\end{frame}

\begin{frame}{Größter gemeinsamer Teiler}
    \begin{exampleblock}{Programmierung}
        Die \emph{ggt}-Funktion lässt sich so programmieren:
        \begin{align*}
            \mathrm{ggt}\left( a, b\right) = \begin{cases} a &\text{falls } b = 0 \\ \mathrm{ggt}\left( b, a\mod b\right)& \end{cases}
        \end{align*}
    \end{exampleblock}
\end{frame}

\section{Algorithmen}

\begin{frame}{Algorithmen}
    \begin{block}{Eigenschaften}
        Ein Algorithmus\dots
        \begin{itemize}
            \item hat eine endliche Beschreibung, 
                \pause
            \item besteht aus elementaren Aussagen,
                \pause
            \item ist deterministisch,
                \pause
            \item gibt endliche Ausgabe auf endliche Eingabe aus,
                \pause
            \item hat endlich viele Schritte,
                \pause
            \item ist skalierbar
                \pause
            \item und ist nachvollziehbar
        \end{itemize}
    \end{block}
\end{frame}

\begin{frame}{Schleifen}
    \begin{block}{Arten}
        \begin{description}
            \item[while] Wiederholen, wenn eine Bedingung erfüllt ist.
                \pause
            \item[for] $n$-Mal wiederholen.
                \pause
            \item[do-while] Wiederholen, danach nochmal, wenn eine Bedingung erfüllt ist.
        \end{description}
    \end{block}
\end{frame}

\begin{frame}{Schleifen}
    \begin{exampleblock}{Beispiel 1}
        \begin{algorithm}
            \begin{algorithmic}
                \Require $x \in \mathbb{N}$
                \State $i \gets 0$
                \While{$x > 1$}
                    \State $x\gets x \div2 2$
                    \State $i\gets i + 1$
                \EndWhile
                \Ensure $i$
            \end{algorithmic}
        \end{algorithm}
    \end{exampleblock}
\end{frame}

\begin{frame}{Schleifen}
    \begin{exampleblock}{Beispiel 2}
        \begin{algorithm}
            \begin{algorithmic}
                \State $k \gets 0$
                \For{i}{0}{20}
                    \State $k\gets i$
                \EndFor
                \Ensure $k$
            \end{algorithmic}
        \end{algorithm}
    \end{exampleblock}
\end{frame}

\begin{frame}{Schleifen}
    \begin{exampleblock}{Beispiel 3}
        Gegeben sei ein Wort $w$ der Länge $\left| w\right| = n$. Das Array $W$ hat an $i$-ter Stelle den $i$-ten Buchstabe von $w$. $w$ ist $\epsilon$-frei.
        \begin{algorithm}
            \begin{algorithmic}
                \State $c \gets 0$
                \For{i}{0}{n-1}
                \State $c\gets \begin{cases} c + 1& \text{falls } W\left[ i\right] = x \\ c
                    &\text{sonst}\end{cases}$
                \EndFor
                \Ensure $c$
            \end{algorithmic}
        \end{algorithm}
    \end{exampleblock}
\end{frame}

\begin{frame}{Schleifen}
    \begin{exampleblock}{Übung 1, Winter 2008/2009}
        Es sei $A$ ein Alphabet.\\
        Schreiben Sie einen Algorithmus auf, der folgendes leistet: Als Eingaben erhält er ein Wort $w$ über $A$ und zwei Symbole $x \in A$ und $y \in A$. Am Ende soll eine Variable $r$ den Wert $0$ oder $1$ haben, und zwar soll gelten:\\
        \begin{align*}
            r = \begin{cases} 1 & \text{falls irgendwo in w direkt hintereinander erst $x$ dann $y$ vorkommt}\\ 0 &\text{sonst}\end{cases}
        \end{align*}
        Benutzen Sie zum Zugriff auf das $i$-te Symbol von $w$ die Schreibweise $w\left( i\right)$. Formulieren Sie den Algorithmus mit Hilfe einer for-Schleife.
    \end{exampleblock}
\end{frame}

\section{Schleifeninvarianzen}

\begin{frame}{Schleifeninvariante}
    \begin{block}{Definition}
        Eine Schleifeninvariante ist eine Eigenschaft einer Schleife, die bei jedem Schleifenzeitpunkt gültig ist.
    \end{block}
    \pause
    \begin{block}{Hä?}
        Eine Schleifeninvariante ist zum Beispiel
        \begin{itemize}
            \item ein Wertebereich für eine Variable oder
            \item ein Verhältnis zweiter Variablen.
        \end{itemize}
    \end{block}
\end{frame}
\begin{frame}{Schleifeninvarianzen}
    \begin{block}{Wofür?}
        Mit Schleifeninvarianten lassen sich Algorithmen überprüfen.
    \end{block}
    \pause
    \begin{block}{Wie?}
        Mit vollständiger Induktion :-)
    \end{block}
\end{frame}
\begin{frame}{Schleifeninvarianzen}
    \begin{block}{Beispiel}
        \begin{algorithm}
            \begin{algorithmic}
                \Require $a, b \in \mathbb{N}_0$
                \State $S \gets a$
                \State $Y \gets b$
                \For{i}{0}{b-1}
                    \State $ S \gets S - 1$
                    \State $Y \gets Y -1$
                \EndFor
                \Ensure $S$
            \end{algorithmic}
        \end{algorithm}
    \end{block}
    \pause
    \begin{exampleblock}{Übung}
        Algorithmus mit $a = 3$ und $b = 4$ ausprobieren und Werte für $S$ und $Y$ bei jedem Schleifendurchlauf finden.
    \end{exampleblock}
\end{frame}
\begin{frame}{Schleifeninvarianzen}
    \begin{exampleblock}{Winter 2008/2009}
        \begin{columns}
            \column{.60\textwidth}
                \begin{algorithm}
                    \begin{algorithmic}
                        \Require $a, b \in \mathbb{N}_0$
                        \State $X_0 \gets a$
                        \State $Y_0 \gets b$
                        \State $P_0 \gets 1$
                        \State $Z_0 \gets X_0 \mod 2$
                        \State $n \gets 1 + \left[ \log_2 a\right]$
                        \For{i}{0}{n-1}
                            \State $P_\mathrm{i+1} \gets P_\mathrm{i}\cdot Y_\mathrm{i}^\mathrm{Z_\mathrm{i}}$
                            \State $X_\mathrm{i+1} \gets X_\mathrm{i} \div 2$
                            \State $Y_\mathrm{i-1} \gets Y_\mathrm{i}^2$
                            \State $Z_\mathrm{i+1} \gets X_\mathrm{i+1} \mod 2$
                        \EndFor
                    \end{algorithmic}
                \end{algorithm}
            \column{.40\textwidth}
                Beweisen Sie durch vollständige Induktion über $i$ die Schleifeninvariante:
                \begin{align*}
                    \forall i \in \mathbb{N}_0: P_\mathrm{i} \cdot Y_\mathrm{i}^\mathrm{X_\mathrm{i}} = b^a
            \end{align*}
        \end{columns}
    \end{exampleblock}
\end{frame}

\end{document}
