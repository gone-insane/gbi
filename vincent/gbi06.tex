%% LaTeX-Beamer template for KIT design
%% by Erik Burger, Christian Hammer
%% title picture by Klaus Krogmann
%%
%% version 2.1
%%
%% mostly compatible to KIT corporate design v2.0
%% http://intranet.kit.edu/gestaltungsrichtlinien.php
%%
%% Problems, bugs and comments to
%% burger@kit.edu

\documentclass[18pt]{beamer}
\usepackage[utf8x]{inputenc}
\usepackage{units}
\usepackage{booktabs}

%% CUSTOM
\usepackage{amsmath}
\usepackage{algpseudocode}

%% Definitions
\DeclareMathOperator{\div2}{div}
\renewcommand{\algorithmicrequire}{\textbf{Input:}}
\renewcommand{\algorithmicensure}{\textbf{Output:}}
\algnewcommand\algorithmicto{\textbf{to}}
\algrenewtext{For}[3]{\algorithmicfor\ $#1 \gets #2$ \algorithmicto\ $#3$ \algorithmicdo}
\algnewcommand\algorithmicod{\textbf{od}}
\algrenewtext{EndWhile}{\algorithmicod}
\algrenewtext{EndFor}{\algorithmicod}
%\AtBeginSection[]{%
%\begin{frame}<beamer> % do nothing in handouts
%    \frametitle{Überblick}
%    \tableofcontents[sectionstyle=show/shaded,
%    subsectionstyle=show/show/hide]
%\end{frame}
%}
%\AtBeginSubsection[]{%
%\begin{frame}<beamer> % do nothing in handouts
%    \frametitle{Überblick}
%    \tableofcontents[sectionstyle=show/shaded,
%    subsectionstyle=show/shaded/hide]
%\end{frame}
%}

%% SLIDE FORMAT

% use 'beamerthemekit' for standard 4:3 ratio
% for widescreen slides (16:9), use 'beamerthemekitwide'

\usepackage{templates/beamerthemekit}
%\usepackage{templates/beamerthemekitwide}

 %% TITLE PICTURE

 % if a custom picture is to be used on the title page, copy it into the 'logos'
 % directory, in the line below, replace 'mypicture' with the 
 % filename (without extension) and uncomment the following line
 % (picture proportions: 63 : 20 for standard, 169 : 40 for wide
 % *.eps format if you use latex+dvips+ps2pdf, 
 % *.jpg/*.png/*.pdf if you use pdflatex)


 \titleimage{banner}
 
 
%% Define some colors:
\definecolor{darkblue}{rgb}{0,0,.5}
\definecolor{darkgreen}{rgb}{0,.5,0}

 %% TITLE LOGO

 % for a custom logo on the front page, copy your file into the 'logos'
 % directory, insert the filename in the line below and uncomment it

\titlelogo{logo_150x150}
 
 % (*.eps format if you use latex+dvips+ps2pdf,
 % *.jpg/*.png/*.pdf if you use pdflatex)
 
 %% TikZ INTEGRATION
 
 % use these packages for PCM symbols and UML classes
 % \usepackage{templates/tikzkit}
 % \usepackage{templates/tikzuml}
 
 % the presentation starts here
 
\author{Dominik Muth - dominik.muth@student.kit.edu}
\institute{Institut f\"ur Informatik}

\subtitle{Foliensatz 6}
\date{29. November 2012}

\begin{document}

\begin{frame}
    \titlepage
\end{frame}

\begin{frame}{Outline/Gliederung}
    \tableofcontents
\end{frame}

\section{Übungsblatt 5}
\begin{frame}{Allgemeine Fehler, Fragen}
    \begin{block}{Allgemeines}
        \begin{itemize}
            \item Randbedingungen bei Sprachen überprüfen ($\varepsilon$)
            \item ``Für ein beliebiges aber festes\dots ''
        \end{itemize}
    \end{block}
\end{frame}

\section{Wiederholung} 
\begin{frame} {Wiederholung - Quiz}
    \begin{itemize}
        \item $\neg (A \land B) \Leftrightarrow \neg A \lor \neg B$
        \only<2-> {\color{darkgreen}$\surd$}\\
        \color{black}
        
        \item $x^2$ ist eine surjektive Abbildung
        \only<3-> {\color{darkgreen}$\surd$\color{red}$X$}\\
        \color{black}
    \end{itemize}
\end{frame}
\begin{frame}{Wiederholung}
    \begin{exampleblock}{Aufgaben}
        \begin{itemize}
            \item Was ist mit $\left( f\circ g \right)\left( x \right)$ gemeint?
            \item Wann ist eine Funktion injektiv?
            \item Wann ist eine Funktion surjektiv?
        \end{itemize}
    \end{exampleblock}
\end{frame}
\begin{frame} {Wiederholung - Aufgaben}
    \begin{block}{Binäre Operationen}
        Geben Sie für folgende aussagenlogische Formeln jeweils einen 
        arithmetischen Ausdruck an, so dass das Ergebnis 
        den Wahrheitswerten der aussagenlogischen Formel entspricht. 
        Verwenden Sie für den Ausdruck nur die Operatoren +, - und 
        $\cdot$ sowie konstante Zahlen. 0 bzw. 1 repräsentiert dabei den
         Wahrheitswert $falsch$ bzw. $wahr$.
         
         \begin{itemize}
            \item $ A \lor B$
            
            \item $ A \Rightarrow B $
            
            \item $ A \Leftrightarrow B $
         \end{itemize}
    \end{block}
\end{frame}
\begin{frame}{Zum warm werden}
    \begin{block}{Vollständige Induktion}
        Beweisen sie folgende Gleichung durch vollständige Induktion:\\
        \[ \sum_{i=1}^{n}(2i-1) = n^2 \]
    \end{block}
\end{frame}

\section{Übersetzungen}
\begin{frame}{Definition}
    \begin{block}{Definition von $\Num2$}
        Zu einer Zahlenbasis $b$ gilt
        \begin{align*}
            \Num2_b\left( \varepsilon \right) &= 0\\
            \forall w \in Z_b^*: \forall x \in Z_b: \Num2_b\left( wx \right) &= b\cdot \Num2_b\left( w \right) + \Num2_b\left( x \right)
        \end{align*}
    \end{block}
    \begin{exampleblock}{Beispiel}
        Gegeben ist die Zahl $101$ im Binärsystem ($b=2$). Umrechung:
        \begin{align*}
            \Num2_2\left( 101 \right) &= 2\cdot\Num2_2\left( 10 \right) &+ \Num2_2\left( 1 \right)&\\
                &= 2\cdot\Num2_2\left( 10 \right) &+ 1&\\
                &= 2\cdot\left( 2\cdot\Num2_2\left( 1 \right) + \Num2_2\left( 0 \right) \right) &+ 1&\\
                &= 2\cdot\left( 2\cdot 1 + 0 \right) &+ 1&\\
                &= 2\cdot 2 + 1 &&= 5
        \end{align*}
    \end{exampleblock}
\end{frame}
\begin{frame}{Übung}
    \begin{exampleblock}{Jetzt ihr}
        \begin{align*}
            \Num2_2\left( 1010 \right) &= \visible<2->{2\cdot\Num2_2\left(  101\right) + \Num2_2\left( 0 \right)\\
                &= \dots = 10}\\
            \Num2_4\left( 321 \right) &= \visible<3->{4\cdot\Num2_4\left( 32 \right) + \Num2_4\left( 1 \right)}\\
                &\visible<4->{=4\cdot\left( 4\cdot\Num2_4\left( 3 \right) + \Num2_4\left( 2 \right) \right) + \Num2_4\left( 1 \right)}\\
                &\visible<5->{=57}\\
                \Num2_{16}\left( B2 \right) &= \visible<6->{16\cdot\Num2_{16}\left( B \right) + \Num2_{16}\left( 2 \right)}\\
                    \visible<7->{&= 16\cdot 11 + 2}\\
                    \visible<8->{&= 178}
        \end{align*}
    \end{exampleblock}
\end{frame}
\begin{frame}{Übung}
    \begin{exampleblock}{Regeln}
        \begin{align*}
            \Num2_2\left( 11 \right) &= \visible<2->{3}\\
            \Num2_2\left( 111 \right) &= \visible<3->{7}\\
            \Num2_2\left( 1111 \right) &= \visible<4->{15}
        \end{align*}
        Gibt es eine Regel? \visible<5->{$\Num2_2\left( 1^m \right) = 2^m - 1$}\\
        Geht das auch in anderen Fällen?
        \begin{align*}
            \Num2_3\left( 22 \right) &= \visible<6->{8}\\
            \Num2_3\left( 222 \right) &= \visible<7->{26}\\
            \Num2_3\left( 2222 \right) &= \visible<8->{80}
        \end{align*}
    \end{exampleblock}
\end{frame}
\begin{frame}{Als Algorithmus}
    \begin{exampleblock}{Von Binär nach Dezimal}
        \visible<2->{
        \begin{algorithm}
            \begin{algorithmic}
                \Require $w \in \mathbb{Z_2}^*$
                \State $x \gets 0$
                \For{i}{0}{|w| - 1}
                    \State $x$ \gets $2\, x + \Num2_2 \left( w\left( i \right) \right)$
                \EndFor
                \Ensure $x$
            \end{algorithmic}
        \end{algorithm}}
    \end{exampleblock}
\end{frame}
\begin{frame}{Aufgabe}
    \begin{exampleblock}{Winter 2010/2011}
        Es bezeichne $\mathbb{Z}$ die Menge der ganzen Zahlen. Gegeben sei eine Ziffernmenge $Z_{-2} = \left\{ N, E \right\}$ mit der Festlegung $\Num2_2\left( N \right) = 0$ und $\Num2_2\left( E \right) = 1$. Wir definieren eine Abbildung $\Num2_{-2}: Z_{-2}^* \rightarrow \mathbb{Z}$ wie folgt:
        \begin{align*}
            \Num2_{-2}\left( \varepsilon \right) &= 0\\
            \forall w \in Z_{-2}^*: \forall x \in Z_{-2}: \Num2_{-2}\left( wx \right) &= -2\cdot \Num2_{-2}\left( w \right) + \Num2_{-2}\left( x \right)
        \end{align*}
        \begin{itemize}
            \item Geben Sie für $w \in \left\{ E, EN, EE, ENE, EEN, EEE\right\}$ jeweils $\Num2_{-2}\left( w \right)$ an.
            \item Für welche Zahlen $x\in\mathbb{Z}$ gibt es ein $w \in Z_{-2}^*$ mit $\Num2_{-2}\left( w \right) = x$?
            \item Wie kann man an einem Wort $w \in Z_{-2}^*$ erkennen, ob $\Num2_{-2}\left( w \right)$ negativ, Null oder positiv ist?
        \end{itemize}
    \end{exampleblock}
\end{frame}

\section{Homomorphismus}
\begin{frame}{Definition}
    \begin{block}{Definition}
        $A$ und $B$ seien Alphabete. $h$ ist eine Abbildung: $h: A\rightarrow B^*$ und $h^{**}: A^*\rightarrow B^*$. \\ Es muss gelten, dass $h^{**}$ ein Homomorphismus ist:
    \pause
        \begin{align*}
            h^{**} &= \varepsilon\\
            \forall w \in A^*: \forall x \in A: h^{**}\left( wx \right) &= h^{**}\left( w \right)h\left( x \right)
        \end{align*}
    \end{block}
\end{frame}
\begin{frame}{Homomorphismus}
    \begin{block}{Eigenschaften}
        Ein Homomorphismus ist\dots
        \begin{itemize}
            \item Strukturerhaltend\\
                $\forall xy \in A^*: h\left( xy \right) = h\left( x \right)\circ h\left( y \right)$\pause
            \item $\varepsilon$-frei, falls\\
                $\forall x \in A: h\left( x \right) \neq \varepsilon$\pause
            \item Präfixfrei, falls\\
                $\forall w \in A^*: \nexists v,z \in A^* \wedge w \neq vz: h\left( w \right) = h\left( v \right) \circ h\left( z \right)$\\
                Anschaulich: Für verschiedene $x_1$ und $x_2$ gilt: $h\left( x_1 \right)$ ist kein Präfix von $h\left( x_2 \right)$.
        \end{itemize}
    \end{block}
\end{frame}
\begin{frame}{Homomorphismus}
    \begin{exampleblock}{Beispiel}
        $h: \left\{ a,b,c \right\}^* \rightarrow \left\{ 0,1 \right\}^*$. Es ist
        \begin{itemize}
            \item $h\left( a \right) = 1$
            \item $h\left( b \right) = 01$
            \item $h\left( c \right) = 001$
        \end{itemize}
        \pause
        Ist der Homomorphismus Präfixfrei? Ist er $\varepsilon$-frei?
        \pause
        Wir erhalten ein beliebiges Codewort und dekodieren dies. Wie?
        \begin{align*}
            u\left( w \right) = \begin{cases} 
                \varepsilon & \text{falls } w = \varepsilon\\
                a\cdot u\left( w' \right) & \text{falls } w = 1w'\\
                b\cdot u\left( w' \right) & \text{falls } w = 01w'\\
                c\cdot u\left( w' \right) & \text{falls } w = 001w'
            \end{cases}
        \end{align*}
        \pause
        Dekodiere $100101$.
    \end{exampleblock}
\end{frame}
\begin{frame}{Beispiel 2}
    Gegeben ist
    \begin{align*}
        h\left( x \right) = \begin{cases}
            2 & \text{falls } x = a\\
            3 & \text{falls } x = b
        \end{cases}
    \end{align*}
    \begin{itemize}
        \item Strukturerhaltend?\\\visible<2->{$h\left( aba \right) = h\left( a \right)\circ h\left( b \right) \circ h\left( a \right) = 232$}
        \item $\varepsilon$-frei: \visible<3->{Annahme $h\left( c \right) = \varepsilon$, $h\left( b \right) = 2$.\\
            Gegeben: Verschlüsseltes Wort und Homomorphismus: $h\left( w \right) = 2$. Woher weiß ich, wieviele $c$ in meinem Wort $w$ sind?}
        \item Präfixfrei: \visible<4->{$h\left( a \right) = 2$, $h\left( b \right) = 3$, $h\left( c \right) = 23$. Woher weiß ich bei $h\left( w \right) = 23$, was $w$ ist?}
    \end{itemize}
\end{frame}
\begin{frame}{Homomorphismen}
    \begin{block}{Wie auf Wörter übertragen?}
        Eigenschaften, welche sich ausnutzen lassen:\\
        \begin{itemize}
            \item Wörter lassen sich konkatenieren
            
            \item Wörter, lassen sich auf Werte abbilden
        \end{itemize}
    \end{block}
    
    \pause		
    
    \begin{exampleblock}{Beispiel}
        -- $h(a) = 001$ und $h(b) = 1101$\\
        -- dann ist $h(bba) = h(b)h(b)h(a) = 1101 \cdot 1101 \cdot 001 = 11011101001$\\
        \vspace{10pt}
        \visible<3->{Warum?}
    \end{exampleblock}
\end{frame}

\begin{frame}{Huffman-Codierung}
    \begin{block}{Beschreibung}
        Die Huffman-Codierung ordnet Wörter bzw. Wörterblöcke durch einen Homomorphismus eine Codierung zu, die umso länger wird, je seltener das Wort vorkommt.
        
    \end{block}
    \begin{exampleblock}{Beispiel}
        Gegeben: $w = analysis$. Vorgehen:
        \begin{enumerate}
            \item Tabelle aufstellen: Wie oft kommt jedes Symbol vor?
            \item Baum aufstellen
            \item Kanten beschriften
        \end{enumerate}
    \end{exampleblock}
\end{frame}
\begin{frame}{Huffman-Codierung}
    \begin{block}{Geht das noch besser?}
        Ja, wenn ein $e$ zum Beispiel oft gefolgt von einem $i$ kommt, kann dies zu einem neuen Zeichen zusammengefasst werden.\\
        Alternativ: Arithmetische Kodierung.
    \end{block}
\end{frame}
\begin{frame}{Aufgaben}
    \begin{exampleblock}{Aufgabe Winter 2008}
        Das Wort
        \begin{align*}
            w = 0000 0001 0011 0001 0011 0000 0000 1110 0001 0000
        \end{align*}
        soll komprimiert werden.
        \begin{itemize}
            \item Zerlegen Sie $w$ in Viererblöcke und bestimmen Sie die Häufigkeit der vorkommenden Blöcke.
            \item Zur Kompression soll ein Huffman-Code verwendet werden. Stellen Sie ienen Baum auf. Beschriften Sie alle Knoten und Kanten
            \item Geben Sie die Codierung des Wortes $w$ mit Ihrem Code an.
        \end{itemize}
        \visible<2->{Lösung: $101 0010 1001 1100 0011$}
    \end{exampleblock}
\end{frame}
\begin{frame}{Aufgaben}
    \begin{exampleblock}{Aufgabe Winter 2010}
        Seien $n,k \in \mathbb{N}_0$ mit $1\geq k \geq n$. In einem Wort $w\in \left\{ a,b,c \right\}^*$ der Länge $3n$ komme $k$ mal das Zeichen $a$, $n$ mal das Zeichen $b$ und $2n-k$ mal das Zeichen $c$ vor.
        \begin{itemize}
            \item Geben Sie den für die Huffman-Codierung benötigten Baum an
            \item Geben Sie in Abhängigkeit von k und n die Länge des zu w gehörenden Huffman-Codes an
        \end{itemize}
        \visible<2->{Lösung Teil b: $2k + 2n+2n-k = 4n+k$}
    \end{exampleblock}
\end{frame}
\end{document}
